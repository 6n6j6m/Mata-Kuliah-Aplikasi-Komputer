\documentclass{article}

\usepackage{eumat}

\begin{document}
\begin{eulernotebook}
\eulerheading{EMT untuk Perhitungan Aljabar}
\begin{eulercomment}
Muhammad Najmi Rahmani (23030630080)

Pada notebook ini Anda belajar menggunakan EMT untuk melakukan
berbagai perhitungan terkait dengan materi atau topik dalam Aljabar.
Kegiatan yang harus Anda lakukan adalah sebagai berikut:

- Membaca secara cermat dan teliti notebook ini;\\
- Menerjemahkan teks bahasa Inggris ke bahasa Indonesia;\\
- Mencoba contoh-contoh perhitungan (perintah EMT) dengan cara
meng-ENTER setiap perintah EMT yang ada (pindahkan kursor ke baris
perintah)\\
- Jika perlu Anda dapat memodifikasi perintah yang ada dan memberikan
keterangan/penjelasan tambahan terkait hasilnya.\\
- Menyisipkan baris-baris perintah baru untuk mengerjakan soal-soal
Aljabar dari file PDF yang saya berikan;\\
- Memberi catatan hasilnya.\\
- Jika perlu tuliskan soalnya pada teks notebook (menggunakan format
LaTeX).\\
- Gunakan tampilan hasil semua perhitungan yang eksak atau simbolik
dengan format LaTeX. (Seperti contoh-contoh pada notebook ini.)

\end{eulercomment}
\eulersubheading{Contoh pertama}
\begin{eulercomment}
Menyederhanakan bentuk aljabar:

\end{eulercomment}
\begin{eulerformula}
\[
6x^{-3}y^5\times -7x^2y^{-9}
\]
\end{eulerformula}
\begin{eulercomment}
\end{eulercomment}
\begin{eulerprompt}
>$&6*x^(-3)*y^5*-7*x^2*y^(-9)
\end{eulerprompt}
\begin{eulerformula}
\[
-\frac{42}{x\,y^4}
\]
\end{eulerformula}
\begin{eulercomment}
Menjabarkan:

\end{eulercomment}
\begin{eulerformula}
\[
(6x^{-3}+y^5)(-7x^2-y^{-9})
\]
\end{eulerformula}
\begin{eulerprompt}
>$&showev('expand((6*x^(-3)+y^5)*(-7*x^2-y^(-9))))
\end{eulerprompt}
\begin{eulerformula}
\[
{\it expand}\left(\left(-\frac{1}{y^9}-7\,x^2\right)\,\left(y^5+
 \frac{6}{x^3}\right)\right)=-7\,x^2\,y^5-\frac{1}{y^4}-\frac{6}{x^3
 \,y^9}-\frac{42}{x}
\]
\end{eulerformula}
\begin{eulercomment}
\end{eulercomment}
\eulersubheading{Baris Perintah}
\begin{eulercomment}
Baris perintah Euler terdiri dari satu atau beberapa perintah Euler
yang diikuti oleh titik koma ";" atau koma ",". Titik koma mencegah
pencetakan hasil. Koma setelah perintah terakhir dapat dihilangkan.


Baris perintah berikut hanya akan mencetak hasil ekspresi, bukan
perintah penugasan atau format.
\end{eulercomment}
\begin{eulerprompt}
>r:=2; h:=4; pi*r^2*h/3
\end{eulerprompt}
\begin{euleroutput}
  16.7551608191
\end{euleroutput}
\begin{eulercomment}
Perintah harus dipisahkan dengan spasi. Baris perintah berikut
mencetak dua hasilnya.
\end{eulercomment}
\begin{eulerprompt}
>pi*2*r*h, %+2*pi*r*h // Ingat tanda % menyatakan hasil perhitungan terakhir sebelumnya
\end{eulerprompt}
\begin{euleroutput}
  50.2654824574
  100.530964915
\end{euleroutput}
\begin{eulercomment}
Baris perintah dieksekusi sesuai urutan pengguna menekan tombol enter.
Jadi Anda akan mendapatkan nilai baru setiap kali Anda mengeksekusi
baris kedua.
\end{eulercomment}
\begin{eulerprompt}
>x := 1;
>x := cos(x) // nilai cosinus (x dalam radian)
\end{eulerprompt}
\begin{euleroutput}
  0.540302305868
\end{euleroutput}
\begin{eulerprompt}
>x := cos(x)
\end{eulerprompt}
\begin{euleroutput}
  0.857553215846
\end{euleroutput}
\begin{eulercomment}
Jika dua baris dihubungkan dengan "..." kedua baris akan selalu
dieksekusi secara bersamaan.
\end{eulercomment}
\begin{eulerprompt}
>x := 1.5; ...
>x := (x+2/x)/2, x := (x+2/x)/2, x := (x+2/x)/2, 
\end{eulerprompt}
\begin{euleroutput}
  1.41666666667
  1.41421568627
  1.41421356237
\end{euleroutput}
\begin{eulercomment}
Ini juga merupakan cara yang baik untuk menyebarkan perintah yang
panjang ke dua atau lebih baris. Anda dapat menekan Ctrl+Return untuk
membagi baris menjadi dua pada posisi kursor saat ini, atau Ctlr+Back
untuk menggabungkan baris-baris tersebut.


Untuk melipat semua baris yang terdiri dari beberapa baris, tekan
Ctrl+L. Kemudian baris-baris berikutnya hanya akan terlihat, jika
salah satunya menjadi fokus. Untuk melipat satu baris yang terdiri
dari beberapa baris, mulailah baris pertama dengan "\%+ ".
\end{eulercomment}
\begin{eulerprompt}
>%+ x=4+5; ...
\end{eulerprompt}
\begin{eulercomment}
Baris yang dimulai dengan \%\% tidak akan terlihat sama sekali.
\end{eulercomment}
\begin{euleroutput}
  81
\end{euleroutput}
\begin{eulercomment}
Euler mendukung perulangan dalam baris perintah, asalkan dapat
dimasukkan ke dalam satu baris atau beberapa baris. Dalam program,
pembatasan ini tentu saja tidak berlaku. Untuk informasi lebih lanjut,
lihat pengantar berikut.
\end{eulercomment}
\begin{eulerprompt}
>x=1; for i=1 to 5; x := (x+2/x)/2, end; // menghitung akar 2
\end{eulerprompt}
\begin{euleroutput}
  1.5
  1.41666666667
  1.41421568627
  1.41421356237
  1.41421356237
\end{euleroutput}
\begin{eulercomment}
Tidak apa-apa menggunakan beberapa baris. Pastikan baris diakhiri
dengan "...".
\end{eulercomment}
\begin{eulerprompt}
>x := 1.5; // comments go here before the ...
>repeat xnew:=(x+2/x)/2; until xnew~=x; ...
>   x := xnew; ...
>end; ...
>x,
\end{eulerprompt}
\begin{euleroutput}
  1.41421356237
\end{euleroutput}
\begin{eulercomment}
Struktur kondisional juga berfungsi.
\end{eulercomment}
\begin{eulerprompt}
>if E^pi>pi^E; then "Thought so!", endif;
\end{eulerprompt}
\begin{euleroutput}
  Thought so!
\end{euleroutput}
\begin{eulercomment}
Saat Anda menjalankan perintah, kursor dapat berada di posisi mana pun
di baris perintah. Anda dapat kembali ke perintah sebelumnya atau
melompat ke perintah berikutnya dengan tombol panah. Atau Anda dapat
mengklik bagian komentar di atas perintah untuk membuka perintah
tersebut.

Saat Anda menggerakkan kursor di sepanjang baris, pasangan tanda
kurung buka dan tutup akan disorot. Perhatikan juga baris status.
Setelah tanda kurung buka fungsi sqrt(), baris status akan menampilkan
teks bantuan untuk fungsi tersebut. Jalankan perintah dengan tombol
return.
\end{eulercomment}
\begin{eulerprompt}
>sqrt(sin(10°)/cos(20°))
\end{eulerprompt}
\begin{euleroutput}
  0.429875017772
\end{euleroutput}
\begin{eulercomment}
Untuk melihat bantuan untuk perintah terbaru, buka jendela bantuan
dengan F1. Di sana, Anda dapat memasukkan teks untuk dicari. Pada
baris kosong, bantuan untuk jendela bantuan akan ditampilkan. Anda
dapat menekan escape untuk menghapus baris, atau untuk menutup jendela
bantuan.

Anda dapat mengklik dua kali pada perintah apa pun untuk membuka
bantuan untuk perintah ini. Coba klik dua kali perintah exp di bawah
ini pada baris perintah.
\end{eulercomment}
\begin{eulerprompt}
>exp(log(2.5))
\end{eulerprompt}
\begin{euleroutput}
  2.5
\end{euleroutput}
\begin{eulercomment}
Anda juga dapat menyalin dan menempel di Euler. Gunakan Ctrl-C dan
Ctrl-V untuk ini. Untuk menandai teks, seret tetikus atau gunakan
shift bersamaan dengan tombol kursor apa pun. Selain itu, Anda dapat
menyalin tanda kurung yang disorot.
\end{eulercomment}
\eulersubheading{Sintaks Dasar}
\begin{eulercomment}
Euler mengetahui fungsi matematika yang umum. Seperti yang telah Anda
lihat di atas, fungsi trigonometri bekerja dalam radian atau derajat.
Untuk mengonversi ke derajat, tambahkan simbol derajat (dengan tombol
F7) ke nilai, atau gunakan fungsi rad(x). Fungsi akar kuadrat disebut
sqrt di Euler. Tentu saja, x\textasciicircum{}(1/2) juga memungkinkan.

Untuk mengatur variabel, gunakan "=" atau ":=". Demi kejelasan,
pengantar ini menggunakan bentuk yang terakhir. Spasi tidak menjadi
masalah. Namun, spasi di antara perintah diharapkan.

Beberapa perintah dalam satu baris dipisahkan dengan "," atau ";".
Titik koma menghilangkan keluaran perintah. Di akhir baris perintah,
"," diasumsikan, jika ";" tidak ada.
\end{eulercomment}
\begin{eulerprompt}
>g:=9.81; t:=2.5; 1/2*g*t^2
\end{eulerprompt}
\begin{euleroutput}
  30.65625
\end{euleroutput}
\begin{eulercomment}
EMT menggunakan sintaks pemrograman untuk ekspresi. Untuk memasukkan

\end{eulercomment}
\begin{eulerformula}
\[
e^2 \cdot \left( \frac{1}{3+4 \log(0.6)}+\frac{1}{7} \right)
\]
\end{eulerformula}
\begin{eulercomment}
Anda harus menetapkan tanda kurung yang benar dan menggunakan / untuk
pecahan. Perhatikan tanda kurung yang disorot untuk bantuan.
Perhatikan bahwa konstanta Euler e diberi nama E dalam EMT.
\end{eulercomment}
\begin{eulerprompt}
>E^2*(1/(3+4*log(0.6))+1/7)
\end{eulerprompt}
\begin{euleroutput}
  8.77908249441
\end{euleroutput}
\begin{eulercomment}
Untuk menghitung ekspresi rumit seperti

\end{eulercomment}
\begin{eulerformula}
\[
\left(\frac{\frac17 + \frac18 + 2}{\frac13 + \frac12}\right)^2 \pi
\]
\end{eulerformula}
\begin{eulercomment}
Anda perlu memasukkannya dalam bentuk baris.
\end{eulercomment}
\begin{eulerprompt}
>((1/7 + 1/8 + 2) / (1/3 + 1/2))^2 * pi
\end{eulerprompt}
\begin{euleroutput}
  23.2671801626
\end{euleroutput}
\begin{eulercomment}
Letakkan tanda kurung di sekitar sub-ekspresi yang perlu dihitung
terlebih dahulu dengan hati-hati. EMT membantu Anda dengan menyorot
ekspresi yang diakhiri tanda kurung tutup. Anda juga harus memasukkan
nama "pi" untuk huruf Yunani pi.

Hasil perhitungan ini adalah angka floating point. Secara default,
angka ini dicetak dengan akurasi sekitar 12 digit. Pada baris perintah
berikut, kita juga mempelajari cara merujuk ke hasil sebelumnya dalam
baris yang sama.
\end{eulercomment}
\begin{eulerprompt}
>1/3+1/7, fraction %
\end{eulerprompt}
\begin{euleroutput}
  0.47619047619
  10/21
\end{euleroutput}
\begin{eulercomment}
Perintah Euler dapat berupa ekspresi atau perintah primitif. Ekspresi
terdiri dari operator dan fungsi. Jika perlu, ekspresi harus berisi
tanda kurung untuk memaksakan urutan eksekusi yang benar. Jika ragu,
sebaiknya gunakan tanda kurung. Perhatikan bahwa EMT menampilkan tanda
kurung buka dan tutup saat mengedit baris perintah.
\end{eulercomment}
\begin{eulerprompt}
>(cos(pi/4)+1)^3*(sin(pi/4)+1)^2
\end{eulerprompt}
\begin{euleroutput}
  14.4978445072
\end{euleroutput}
\begin{eulercomment}
Operator numerik Euler meliputi

+ unary atau operator plus\\
- unary atau operator minus\\
*, /\\
. perkalian matriks\\
a\textasciicircum{}b pangkat untuk a positif atau integer b (a**b juga berfungsi)\\
n! operator faktorial

dan masih banyak lagi.

Berikut ini beberapa fungsi yang mungkin Anda perlukan. Masih banyak
lagi.

sin,cos,tan,atan,asin,acos,rad,deg\\
log,exp,log10,sqrt,logbase\\
bin,logbin,logfac,mod,floor,ceil,round,abs,sign\\
conj,re,im,arg,conj,real,complex\\
beta,betai,gamma,complexgamma,ellrf,ellf,ellrd,elle\\
bitand,bitor,bitxor,bitnot

Beberapa perintah memiliki alias, misalnya ln untuk log.
\end{eulercomment}
\begin{eulerprompt}
>ln(E^2), arctan(tan(0.5))
\end{eulerprompt}
\begin{euleroutput}
  2
  0.5
\end{euleroutput}
\begin{eulerprompt}
>sin(30°)
\end{eulerprompt}
\begin{euleroutput}
  0.5
\end{euleroutput}
\begin{eulercomment}
Pastikan untuk menggunakan tanda kurung (kurung bundar), jika ada
keraguan tentang urutan eksekusi! Berikut ini tidak sama dengan
(2\textasciicircum{}3)\textasciicircum{}4, yang merupakan default untuk 2\textasciicircum{}3\textasciicircum{}4 dalam EMT (beberapa sistem
numerik melakukannya dengan cara lain).
\end{eulercomment}
\begin{eulerprompt}
>2^3^4, (2^3)^4, 2^(3^4)
\end{eulerprompt}
\begin{euleroutput}
  2.41785163923e+24
  4096
  2.41785163923e+24
\end{euleroutput}
\eulersubheading{Bilangan Riil}
\begin{eulercomment}
Tipe data utama dalam Euler adalah bilangan riil. Bilangan riil
direpresentasikan dalam format IEEE dengan akurasi sekitar 16 digit
desimal.
\end{eulercomment}
\begin{eulerprompt}
>longest 1/3
\end{eulerprompt}
\begin{euleroutput}
       0.3333333333333333 
\end{euleroutput}
\begin{eulercomment}
Representasi ganda internal membutuhkan 8 byte.
\end{eulercomment}
\begin{eulerprompt}
>printdual(1/3)
\end{eulerprompt}
\begin{euleroutput}
  1.0101010101010101010101010101010101010101010101010101*2^-2
\end{euleroutput}
\begin{eulerprompt}
>printhex(1/3)
\end{eulerprompt}
\begin{euleroutput}
  5.5555555555554*16^-1
\end{euleroutput}
\eulersubheading{String}
\begin{eulercomment}
String dalam Euler didefinisikan dengan "...".
\end{eulercomment}
\begin{eulerprompt}
>"A string can contain anything."
\end{eulerprompt}
\begin{euleroutput}
  A string can contain anything.
\end{euleroutput}
\begin{eulercomment}
String dapat dirangkai dengan \textbar{} atau dengan +. Ini juga berlaku untuk
angka, yang dalam kasus tersebut diubah menjadi string.
\end{eulercomment}
\begin{eulerprompt}
>"The area of the circle with radius " + 2 + " cm is " + pi*4 + " cm^2."
\end{eulerprompt}
\begin{euleroutput}
  The area of the circle with radius 2 cm is 12.5663706144 cm^2.
\end{euleroutput}
\begin{eulercomment}
Fungsi cetak juga mengonversi angka menjadi string. Fungsi ini dapat
mengambil sejumlah digit dan sejumlah tempat (0 untuk output yang
padat), dan optimalnya satu unit.
\end{eulercomment}
\begin{eulerprompt}
>"Golden Ratio : " + print((1+sqrt(5))/2,5,0)
\end{eulerprompt}
\begin{euleroutput}
  Golden Ratio : 1.61803
\end{euleroutput}
\begin{eulercomment}
Ada string khusus none, yang tidak dicetak. String ini dikembalikan
oleh beberapa fungsi, ketika hasilnya tidak penting. (Dikembalikan
secara otomatis, jika fungsi tersebut tidak memiliki pernyataan
return.)
\end{eulercomment}
\begin{eulerprompt}
>none
\end{eulerprompt}
\begin{eulercomment}
Untuk mengubah string menjadi angka, cukup evaluasi string tersebut.
Ini juga berlaku untuk ekspresi (lihat di bawah)
\end{eulercomment}
\begin{eulerprompt}
>"1234.5"()
\end{eulerprompt}
\begin{euleroutput}
  1234.5
\end{euleroutput}
\begin{eulercomment}
Untuk mendefinisikan vektor string, gunakan notasi vektor [...]
\end{eulercomment}
\begin{eulerprompt}
>v:=["affe","charlie","bravo"]
\end{eulerprompt}
\begin{euleroutput}
  affe
  charlie
  bravo
\end{euleroutput}
\begin{eulercomment}
Vektor string kosong dilambangkan dengan [none]. Vektor string dapat
dirangkai.
\end{eulercomment}
\begin{eulerprompt}
>w:=[none]; w|v|v
\end{eulerprompt}
\begin{euleroutput}
  affe
  charlie
  bravo
  affe
  charlie
  bravo
\end{euleroutput}
\begin{eulercomment}
String dapat berisi karakter Unicode. Secara internal, string ini
berisi kode UTF-8. Untuk membuat string seperti itu, gunakan u"..."
dan salah satu entitas HTML.

String Unicode dapat dirangkai seperti string lainnya.
\end{eulercomment}
\begin{eulerprompt}
>u"&alpha; = " + 45 + u"&deg;" // pdfLaTeX mungkin gagal menampilkan secara benar
\end{eulerprompt}
\begin{euleroutput}
  α = 45°
\end{euleroutput}
\begin{eulercomment}
I
\end{eulercomment}
\begin{eulercomment}
Dalam komentar, entitas yang sama seperti alpha;, beta; dll. dapat
digunakan. Ini mungkin merupakan alternatif cepat untuk Latex.
(Rincian lebih lanjut pada komentar di bawah).
\end{eulercomment}
\begin{eulercomment}
Ada beberapa fungsi untuk membuat atau menganalisis string unicode.
Fungsi strtochar() akan mengenali string Unicode dan menerjemahkannya
dengan benar.
\end{eulercomment}
\begin{eulerprompt}
>v=strtochar(u"&Auml; is a German letter")
\end{eulerprompt}
\begin{euleroutput}
  [196,  32,  105,  115,  32,  97,  32,  71,  101,  114,  109,  97,  110,
  32,  108,  101,  116,  116,  101,  114]
\end{euleroutput}
\begin{eulercomment}
Hasilnya adalah vektor angka Unicode. Fungsi kebalikannya adalah
chartoutf().
\end{eulercomment}
\begin{eulerprompt}
>v[1]=strtochar(u"&Uuml;")[1]; chartoutf(v)
\end{eulerprompt}
\begin{euleroutput}
  Ü is a German letter
\end{euleroutput}
\begin{eulercomment}
Fungsi utf() dapat menerjemahkan string dengan entitas dalam variabel
menjadi string Unicode.
\end{eulercomment}
\begin{eulerprompt}
>s="We have &alpha;=&beta;."; utf(s) // pdfLaTeX mungkin gagal menampilkan secara benar
\end{eulerprompt}
\begin{euleroutput}
  We have α=β.
\end{euleroutput}
\begin{eulercomment}
Dimungkinkan juga untuk menggunakan entitas numerik.
\end{eulercomment}
\begin{eulerprompt}
>u"&#196;hnliches"
\end{eulerprompt}
\begin{euleroutput}
  Ähnliches
\end{euleroutput}
\eulersubheading{Nilai Boolean}
\begin{eulercomment}
Nilai Boolean direpresentasikan dengan 1=benar atau 0=salah dalam
Euler. String dapat dibandingkan, seperti halnya angka.
\end{eulercomment}
\begin{eulerprompt}
>2<1, "apel"<"banana"
\end{eulerprompt}
\begin{euleroutput}
  0
  1
\end{euleroutput}
\begin{eulercomment}
"dan" adalah operator "\&\&" dan "atau" adalah operator "\textbar{}\textbar{}", seperti
dalam bahasa C. (Kata "dan" dan "atau" hanya dapat digunakan dalam
kondisi "jika".)
\end{eulercomment}
\begin{eulerprompt}
>2<E && E<3
\end{eulerprompt}
\begin{euleroutput}
  1
\end{euleroutput}
\begin{eulercomment}
Operator Boolean mematuhi aturan bahasa matriks.
\end{eulercomment}
\begin{eulerprompt}
>(1:10)>5, nonzeros(%)
\end{eulerprompt}
\begin{euleroutput}
  [0,  0,  0,  0,  0,  1,  1,  1,  1,  1]
  [6,  7,  8,  9,  10]
\end{euleroutput}
\begin{eulercomment}
Anda dapat menggunakan fungsi nonzeros() untuk mengekstrak elemen
tertentu dari sebuah vektor. Dalam contoh ini, kami menggunakan
kondisional isprime(n).
\end{eulercomment}
\begin{eulerprompt}
>N=2|3:2:99 // N berisi elemen 2 dan bilangan2 ganjil dari 3 s.d. 99
\end{eulerprompt}
\begin{euleroutput}
  [2,  3,  5,  7,  9,  11,  13,  15,  17,  19,  21,  23,  25,  27,  29,
  31,  33,  35,  37,  39,  41,  43,  45,  47,  49,  51,  53,  55,  57,
  59,  61,  63,  65,  67,  69,  71,  73,  75,  77,  79,  81,  83,  85,
  87,  89,  91,  93,  95,  97,  99]
\end{euleroutput}
\begin{eulerprompt}
>N[nonzeros(isprime(N))] //pilih anggota2 N yang prima
\end{eulerprompt}
\begin{euleroutput}
  [2,  3,  5,  7,  11,  13,  17,  19,  23,  29,  31,  37,  41,  43,  47,
  53,  59,  61,  67,  71,  73,  79,  83,  89,  97]
\end{euleroutput}
\eulersubheading{Format Output}
\begin{eulercomment}
Format output default EMT mencetak 12 digit. Untuk memastikan bahwa
kita melihat default, kita mengatur ulang formatnya.
\end{eulercomment}
\begin{eulerprompt}
>defformat; pi
\end{eulerprompt}
\begin{euleroutput}
  3.14159265359
\end{euleroutput}
\begin{eulercomment}
Secara internal, EMT menggunakan standar IEEE untuk angka ganda dengan
sekitar 16 digit desimal. Untuk melihat jumlah digit lengkap, gunakan
perintah "longestformat", atau kami menggunakan operator "longest"
untuk menampilkan hasil dalam format terpanjang.
\end{eulercomment}
\begin{eulerprompt}
>longest pi
\end{eulerprompt}
\begin{euleroutput}
        3.141592653589793 
\end{euleroutput}
\begin{eulercomment}
Berikut adalah representasi heksadesimal internal dari angka ganda.
\end{eulercomment}
\begin{eulerprompt}
>printhex(pi)
\end{eulerprompt}
\begin{euleroutput}
  3.243F6A8885A30*16^0
\end{euleroutput}
\begin{eulercomment}
Format keluaran dapat diubah secara permanen dengan perintah format.
\end{eulercomment}
\begin{eulerprompt}
>format(12,5); 1/3, pi, sin(1)
\end{eulerprompt}
\begin{euleroutput}
      0.33333 
      3.14159 
      0.84147 
\end{euleroutput}
\begin{eulercomment}
Format defaultnya adalah(12).
\end{eulercomment}
\begin{eulerprompt}
>format(12); 1/3
\end{eulerprompt}
\begin{euleroutput}
  0.333333333333
\end{euleroutput}
\begin{eulercomment}
Fungsi seperti "shortestformat", "shortformat", "longformat" bekerja
untuk vektor dengan cara berikut.
\end{eulercomment}
\begin{eulerprompt}
>shortestformat; random(3,8)
\end{eulerprompt}
\begin{euleroutput}
    0.66    0.2   0.89   0.28   0.53   0.31   0.44    0.3 
    0.28   0.88   0.27    0.7   0.22   0.45   0.31   0.91 
    0.19   0.46  0.095    0.6   0.43   0.73   0.47   0.32 
\end{euleroutput}
\begin{eulercomment}
Format default untuk skalar adalah format(12). Namun, ini dapat
diubah.
\end{eulercomment}
\begin{eulerprompt}
>setscalarformat(5); pi
\end{eulerprompt}
\begin{euleroutput}
  3.1416
\end{euleroutput}
\begin{eulercomment}
Fungsi "longestformat" juga mengatur format skalar.
\end{eulercomment}
\begin{eulerprompt}
>longestformat; pi
\end{eulerprompt}
\begin{euleroutput}
  3.141592653589793
\end{euleroutput}
\begin{eulercomment}
Sebagai referensi, berikut adalah daftar format output yang paling
penting.

shortestformat shortformat longformat, longestformat

format(length,digits) goodformat(length)

fracformat(length)

defformat

Keakuratan internal EMT adalah sekitar 16 tempat desimal, yang
merupakan standar IEEE. Angka disimpan dalam format internal ini.

Namun, format output EMT dapat diatur dengan cara yang fleksibel.
\end{eulercomment}
\begin{eulerprompt}
>longestformat; pi,
\end{eulerprompt}
\begin{euleroutput}
  3.141592653589793
\end{euleroutput}
\begin{eulerprompt}
>format(10,5); pi
\end{eulerprompt}
\begin{euleroutput}
    3.14159 
\end{euleroutput}
\begin{eulercomment}
Standarnya adalah defformat().
\end{eulercomment}
\begin{eulerprompt}
>defformat; // default
\end{eulerprompt}
\begin{eulercomment}
Ada operator pendek yang hanya mencetak satu nilai. Operator
"terpanjang" akan mencetak semua digit angka yang valid.
\end{eulercomment}
\begin{eulerprompt}
>longest pi^2/2
\end{eulerprompt}
\begin{euleroutput}
        4.934802200544679 
\end{euleroutput}
\begin{eulercomment}
Ada juga operator pendek untuk mencetak hasil dalam format pecahan.
Kami telah menggunakannya di atas.
\end{eulercomment}
\begin{eulerprompt}
>fraction 1+1/2+1/3+1/4
\end{eulerprompt}
\begin{euleroutput}
  25/12
\end{euleroutput}
\begin{eulercomment}
Karena format internal menggunakan cara biner untuk menyimpan angka,
nilai 0,1 tidak akan terwakili secara tepat. Kesalahannya bertambah
sedikit, seperti yang Anda lihat dalam perhitungan berikut.
\end{eulercomment}
\begin{eulerprompt}
>longest 0.1+0.1+0.1+0.1+0.1+0.1+0.1+0.1+0.1+0.1-1
\end{eulerprompt}
\begin{euleroutput}
   -1.110223024625157e-16 
\end{euleroutput}
\begin{eulercomment}
Namun dengan "longformat" default, Anda tidak akan melihat hal ini.
Demi kenyamanan, output angka yang sangat kecil adalah 0.
\end{eulercomment}
\begin{eulerprompt}
>0.1+0.1+0.1+0.1+0.1+0.1+0.1+0.1+0.1+0.1-1
\end{eulerprompt}
\begin{euleroutput}
  0
\end{euleroutput}
\eulerheading{Ekspresi}
\begin{eulercomment}
String atau nama dapat digunakan untuk menyimpan ekspresi matematika,
yang dapat dievaluasi oleh EMT. Untuk ini, gunakan tanda kurung
setelah ekspresi. Jika Anda ingin menggunakan string sebagai ekspresi,
gunakan konvensi untuk menamainya "fx" atau "fxy", dst. Ekspresi lebih
diutamakan daripada fungsi.

Variabel global dapat digunakan dalam evaluasi.
\end{eulercomment}
\begin{eulerprompt}
>r:=2; fx:="pi*r^2"; longest fx()
\end{eulerprompt}
\begin{euleroutput}
        12.56637061435917 
\end{euleroutput}
\begin{eulercomment}
Parameter ditetapkan ke x, y, dan z dalam urutan tersebut. Parameter
tambahan dapat ditambahkan menggunakan parameter yang ditetapkan.
\end{eulercomment}
\begin{eulerprompt}
>fx:="a*sin(x)^2"; fx(5,a=-1)
\end{eulerprompt}
\begin{euleroutput}
  -0.919535764538
\end{euleroutput}
\begin{eulercomment}
Perhatikan bahwa ekspresi akan selalu menggunakan variabel global,
bahkan jika ada variabel dalam suatu fungsi dengan nama yang sama.
(Jika tidak, evaluasi ekspresi dalam fungsi dapat memberikan hasil
yang sangat membingungkan bagi pengguna yang memanggil fungsi
tersebut.)
\end{eulercomment}
\begin{eulerprompt}
>at:=4; function f(expr,x,at) := expr(x); ...
>f("at*x^2",3,5) // computes 4*3^2 not 5*3^2
\end{eulerprompt}
\begin{euleroutput}
  36
\end{euleroutput}
\begin{eulercomment}
Jika Anda ingin menggunakan nilai lain untuk "at" selain nilai global,
Anda perlu menambahkan "at=value".
\end{eulercomment}
\begin{eulerprompt}
>at:=4; function f(expr,x,a) := expr(x,at=a); ...
>f("at*x^2",3,5)
\end{eulerprompt}
\begin{euleroutput}
  45
\end{euleroutput}
\begin{eulercomment}
Sebagai referensi, kami mencatat bahwa koleksi panggilan (dibahas di
tempat lain) dapat berisi ekspresi. Jadi, kita dapat membuat contoh di
atas sebagai berikut.
\end{eulercomment}
\begin{eulerprompt}
>at:=4; function f(expr,x) := expr(x); ...
>f(\{\{"at*x^2",at=5\}\},3)
\end{eulerprompt}
\begin{euleroutput}
  45
\end{euleroutput}
\begin{eulercomment}
Ekspresi dalam x sering digunakan seperti fungsi.\\
Perlu dicatat bahwa mendefinisikan fungsi dengan nama yang sama
seperti ekspresi simbolik global akan menghapus variabel ini untuk
menghindari kebingungan antara ekspresi simbolik dan fungsi.
\end{eulercomment}
\begin{eulerprompt}
>f &= 5*x;
>function f(x) := 6*x;
>f(2)
\end{eulerprompt}
\begin{euleroutput}
  12
\end{euleroutput}
\begin{eulercomment}
Berdasarkan konvensi, ekspresi simbolik atau numerik harus diberi nama
fx, fxy, dst. Skema penamaan ini tidak boleh digunakan untuk fungsi.
\end{eulercomment}
\begin{eulerprompt}
>fx &= diff(x^x,x); $&fx
\end{eulerprompt}
\begin{eulerformula}
\[
x^{x}\,\left(\log x+1\right)
\]
\end{eulerformula}
\begin{eulercomment}
Bentuk khusus dari suatu ekspresi memperbolehkan variabel apa pun
sebagai parameter tanpa nama untuk evaluasi ekspresi, bukan hanya "x",
"y", dst. Untuk ini, awali ekspresi dengan "@(variabel) ...".

\end{eulercomment}
\begin{eulerprompt}
>"@(a,b) a^2+b^2", %(4,5)
\end{eulerprompt}
\begin{euleroutput}
  @(a,b) a^2+b^2
  41
\end{euleroutput}
\begin{eulercomment}
Hal ini memungkinkan untuk memanipulasi ekspresi dalam variabel lain
untuk fungsi EMT yang memerlukan ekspresi dalam "x".

Cara paling dasar untuk mendefinisikan fungsi sederhana adalah dengan
menyimpan rumusnya dalam ekspresi simbolik atau numerik. Jika variabel
utamanya adalah x, ekspresi tersebut dapat dievaluasi seperti halnya
fungsi.

Seperti yang Anda lihat dalam contoh berikut, variabel global terlihat
selama evaluasi.
\end{eulercomment}
\begin{eulerprompt}
>fx &= x^3-a*x;  ...
>a=1.2; fx(0.5)
\end{eulerprompt}
\begin{euleroutput}
  -0.475
\end{euleroutput}
\begin{eulercomment}
Semua variabel lain dalam ekspresi dapat ditentukan dalam evaluasi
menggunakan parameter yang ditetapkan.
\end{eulercomment}
\begin{eulerprompt}
>fx(0.5,a=1.1)
\end{eulerprompt}
\begin{euleroutput}
  -0.425
\end{euleroutput}
\begin{eulercomment}
Suatu ekspresi tidak harus simbolis. Hal ini diperlukan jika ekspresi
tersebut mengandung fungsi yang hanya diketahui dalam kernel numerik,
bukan dalam Maxima.

\begin{eulercomment}
\eulerheading{Matematika Simbolis}
\begin{eulercomment}
EMT mengerjakan matematika simbolis dengan bantuan Maxima. Untuk
detailnya, mulailah dengan tutorial berikut, atau telusuri referensi
untuk Maxima. Para ahli di Maxima harus memperhatikan bahwa terdapat
perbedaan sintaksis antara sintaksis asli Maxima dan sintaksis default
ekspresi simbolis dalam EMT.

Matematika simbolis terintegrasi dengan mulus ke dalam Euler dengan \&.
Setiap ekspresi yang dimulai dengan \& adalah ekspresi simbolis.
Ekspresi tersebut dievaluasi dan dicetak oleh Maxima.

Pertama-tama, Maxima memiliki aritmatika "tak terbatas" yang dapat
menangani angka yang sangat besar.
\end{eulercomment}
\begin{eulerprompt}
>$&44!
\end{eulerprompt}
\begin{eulerformula}
\[
2658271574788448768043625811014615890319638528000000000
\]
\end{eulerformula}
\begin{eulercomment}
Dengan cara ini, Anda dapat menghitung hasil yang besar secara tepat.
Mari kita hitung

\end{eulercomment}
\begin{eulerformula}
\[
C(44,10) = \frac{44!}{34! \cdot 10!}
\]
\end{eulerformula}
\begin{eulerprompt}
>$& 44!/(34!*10!) // nilai C(44,10)
\end{eulerprompt}
\begin{eulerformula}
\[
2481256778
\]
\end{eulerformula}
\begin{eulercomment}
Tentu saja, Maxima memiliki fungsi yang lebih efisien untuk ini
(seperti halnya bagian numerik EMT).
\end{eulercomment}
\begin{eulerprompt}
>$binomial(44,10) //menghitung C(44,10) menggunakan fungsi binomial()
\end{eulerprompt}
\begin{eulerformula}
\[
2481256778
\]
\end{eulerformula}
\begin{eulercomment}
Untuk mempelajari lebih lanjut tentang fungsi tertentu, klik dua kali
pada fungsi tersebut. Misalnya, coba klik dua kali pada "\&binomial" di
baris perintah sebelumnya. Ini akan membuka dokumentasi Maxima
sebagaimana disediakan oleh penulis program tersebut.

Anda akan mempelajari bahwa hal berikut juga berfungsi.

\end{eulercomment}
\begin{eulerformula}
\[
C(x,3)=\frac{x!}{(x-3)!3!}=\frac{(x-2)(x-1)x}{6}
\]
\end{eulerformula}
\begin{eulerprompt}
>$binomial(x,3) // C(x,3)
\end{eulerprompt}
\begin{eulerformula}
\[
\frac{\left(x-2\right)\,\left(x-1\right)\,x}{6}
\]
\end{eulerformula}
\begin{eulercomment}
Jika Anda ingin mengganti x dengan nilai tertentu, gunakan "with".
\end{eulercomment}
\begin{eulerprompt}
>$&binomial(x,3) with x=10 // substitusi x=10 ke C(x,3)
\end{eulerprompt}
\begin{eulerformula}
\[
120
\]
\end{eulerformula}
\begin{eulercomment}
Dengan cara itu Anda dapat menggunakan solusi persamaan dalam
persamaan lain.

Ekspresi simbolik dicetak oleh Maxima dalam bentuk 2D. Alasannya
adalah adanya tanda simbolik khusus dalam string.

Seperti yang telah Anda lihat pada contoh sebelumnya dan berikutnya,
jika Anda telah menginstal LaTeX, Anda dapat mencetak ekspresi
simbolik dengan Latex. Jika tidak, perintah berikut akan mengeluarkan
pesan kesalahan.

Untuk mencetak ekspresi simbolik dengan LaTeX, gunakan \textdollar{} di depan \&
(atau Anda dapat menghilangkan \&) sebelum perintah. Jangan jalankan
perintah Maxima dengan \textdollar{}, jika Anda tidak menginstal LaTeX.
\end{eulercomment}
\begin{eulerprompt}
>$(3+x)/(x^2+1)
\end{eulerprompt}
\begin{eulerformula}
\[
\frac{x+3}{x^2+1}
\]
\end{eulerformula}
\begin{eulercomment}
Ekspresi simbolik diurai oleh Euler. Jika Anda memerlukan sintaksis
yang kompleks dalam satu ekspresi, Anda dapat melampirkan ekspresi
tersebut dalam "...". Menggunakan lebih dari satu ekspresi sederhana
dimungkinkan, tetapi sangat tidak disarankan.
\end{eulercomment}
\begin{eulerprompt}
>&"v := 5; v^2"
\end{eulerprompt}
\begin{euleroutput}
  
                                    25
  
\end{euleroutput}
\begin{eulercomment}
Untuk kelengkapan, kami mencatat bahwa ekspresi simbolik dapat
digunakan dalam program, tetapi harus disertakan dalam tanda kutip.
Selain itu, akan jauh lebih efektif untuk memanggil Maxima pada waktu
kompilasi jika memungkinkan.
\end{eulercomment}
\begin{eulerprompt}
>$&expand((1+x)^4), $&factor(diff(%,x)) // diff: turunan, factor: faktor
\end{eulerprompt}
\begin{eulerformula}
\[
x^4+4\,x^3+6\,x^2+4\,x+1
\]
\end{eulerformula}
\begin{eulerformula}
\[
4\,\left(x+1\right)^3
\]
\end{eulerformula}
\begin{eulercomment}
Sekali lagi, \% merujuk pada hasil sebelumnya.

Untuk mempermudah, kami menyimpan solusi ke variabel simbolik.
Variabel simbolik didefinisikan dengan "\&=".
\end{eulercomment}
\begin{eulerprompt}
>fx &= (x+1)/(x^4+1); $&fx
\end{eulerprompt}
\begin{eulerformula}
\[
\frac{x+1}{x^4+1}
\]
\end{eulerformula}
\begin{eulercomment}
Ekspresi simbolik dapat digunakan dalam ekspresi simbolik lainnya.
\end{eulercomment}
\begin{eulerprompt}
>$&factor(diff(fx,x))
\end{eulerprompt}
\begin{eulerformula}
\[
\frac{-3\,x^4-4\,x^3+1}{\left(x^4+1\right)^2}
\]
\end{eulerformula}
\begin{eulercomment}
Input langsung perintah Maxima juga tersedia. Awali baris perintah
dengan "::". Sintaks Maxima disesuaikan dengan sintaks EMT (disebut
"mode kompatibilitas").
\end{eulercomment}
\begin{eulerprompt}
>&factor(20!)
\end{eulerprompt}
\begin{euleroutput}
  
                           2432902008176640000
  
\end{euleroutput}
\begin{eulerprompt}
>::: factor(10!)
\end{eulerprompt}
\begin{euleroutput}
  
                                 8  4  2
                                2  3  5  7
  
\end{euleroutput}
\begin{eulerprompt}
>:: factor(20!)
\end{eulerprompt}
\begin{euleroutput}
  
                          18  8  4  2
                         2   3  5  7  11 13 17 19
  
\end{euleroutput}
\begin{eulercomment}
Jika Anda ahli dalam Maxima, Anda mungkin ingin menggunakan sintaksis
asli Maxima. Anda dapat melakukannya dengan ":::".
\end{eulercomment}
\begin{eulerprompt}
>::: av:g$ av^2;
\end{eulerprompt}
\begin{euleroutput}
  
                                     2
                                    g
  
\end{euleroutput}
\begin{eulerprompt}
>fx &= x^3*exp(x), $fx
\end{eulerprompt}
\begin{euleroutput}
  
                                   3  x
                                  x  E
  
\end{euleroutput}
\begin{eulerformula}
\[
x^3\,e^{x}
\]
\end{eulerformula}
\begin{eulercomment}
Variabel tersebut dapat digunakan dalam ekspresi simbolik lainnya.
Perhatikan bahwa dalam perintah berikut sisi kanan \&= dievaluasi
sebelum penugasan ke Fx.
\end{eulercomment}
\begin{eulerprompt}
>&(fx with x=5), $%, &float(%)
\end{eulerprompt}
\begin{euleroutput}
  
                                       5
                                  125 E
  
\end{euleroutput}
\begin{eulerformula}
\[
125\,e^5
\]
\end{eulerformula}
\begin{euleroutput}
  
                            18551.64488782208
  
\end{euleroutput}
\begin{eulerprompt}
>fx(5)
\end{eulerprompt}
\begin{euleroutput}
  18551.6448878
\end{euleroutput}
\begin{eulercomment}
Untuk mengevaluasi ekspresi dengan nilai variabel tertentu, Anda dapat
menggunakan operator "with".

Baris perintah berikut juga menunjukkan bahwa Maxima dapat
mengevaluasi ekspresi secara numerik dengan float().
\end{eulercomment}
\begin{eulerprompt}
>&(fx with x=10)-(fx with x=5), &float(%)
\end{eulerprompt}
\begin{euleroutput}
  
                                  10        5
                            1000 E   - 125 E
  
  
                           2.20079141499189e+7
  
\end{euleroutput}
\begin{eulerprompt}
>$factor(diff(fx,x,2))
\end{eulerprompt}
\begin{eulerformula}
\[
x\,\left(x^2+6\,x+6\right)\,e^{x}
\]
\end{eulerformula}
\begin{eulercomment}
Untuk mendapatkan kode Latex untuk suatu ekspresi, Anda dapat
menggunakan perintah tex.
\end{eulercomment}
\begin{eulerprompt}
>tex(fx)
\end{eulerprompt}
\begin{euleroutput}
  x^3\(\backslash\),e^\{x\}
\end{euleroutput}
\begin{eulercomment}
Ekspresi simbolik dapat dievaluasi seperti halnya ekspresi numerik.
\end{eulercomment}
\begin{eulerprompt}
>fx(0.5)
\end{eulerprompt}
\begin{euleroutput}
  0.206090158838
\end{euleroutput}
\begin{eulercomment}
Dalam ekspresi simbolik, ini tidak berfungsi, karena Maxima tidak
mendukungnya. Sebagai gantinya, gunakan sintaks "with" (bentuk yang
lebih baik dari perintah at(...) dari Maxima).
\end{eulercomment}
\begin{eulerprompt}
>$&fx with x=1/2
\end{eulerprompt}
\begin{eulerformula}
\[
\frac{\sqrt{e}}{8}
\]
\end{eulerformula}
\begin{eulercomment}
Penugasan tersebut juga dapat bersifat simbolis.
\end{eulercomment}
\begin{eulerprompt}
>$&fx with x=1+t
\end{eulerprompt}
\begin{eulerformula}
\[
\left(t+1\right)^3\,e^{t+1}
\]
\end{eulerformula}
\begin{eulercomment}
Perintah solve memecahkan ekspresi simbolik untuk variabel dalam
Maxima. Hasilnya adalah vektor solusi.
\end{eulercomment}
\begin{eulerprompt}
>$&solve(x^2+x=4,x)
\end{eulerprompt}
\begin{eulerformula}
\[
\left[ x=\frac{-\sqrt{17}-1}{2} , x=\frac{\sqrt{17}-1}{2} \right] 
\]
\end{eulerformula}
\begin{eulercomment}
Bandingkan dengan perintah numerik "solve" di Euler, yang memerlukan
nilai awal, dan secara opsional nilai target.
\end{eulercomment}
\begin{eulerprompt}
>solve("x^2+x",1,y=4)
\end{eulerprompt}
\begin{euleroutput}
  1.56155281281
\end{euleroutput}
\begin{eulercomment}
Nilai numerik dari solusi simbolik dapat dihitung dengan mengevaluasi
hasil simbolik. Euler akan membaca ulang penugasan x= dst. Jika Anda
tidak memerlukan hasil numerik untuk perhitungan lebih lanjut, Anda
juga dapat membiarkan Maxima menemukan nilai numeriknya.
\end{eulercomment}
\begin{eulerprompt}
>sol &= solve(x^2+2*x=4,x); $&sol, sol(), $&float(sol)
\end{eulerprompt}
\begin{eulerformula}
\[
\left[ x=-\sqrt{5}-1 , x=\sqrt{5}-1 \right] 
\]
\end{eulerformula}
\begin{euleroutput}
  [-3.23607,  1.23607]
\end{euleroutput}
\begin{eulerformula}
\[
\left[ x=-3.23606797749979 , x=1.23606797749979 \right] 
\]
\end{eulerformula}
\begin{eulercomment}
Untuk mendapatkan solusi simbolis yang spesifik, seseorang dapat
menggunakan "with" dan indeks.
\end{eulercomment}
\begin{eulerprompt}
>$&solve(x^2+x=1,x), x2 &= x with %[2]; $&x2
\end{eulerprompt}
\begin{eulerformula}
\[
\left[ x=\frac{-\sqrt{5}-1}{2} , x=\frac{\sqrt{5}-1}{2} \right] 
\]
\end{eulerformula}
\begin{eulerformula}
\[
\frac{\sqrt{5}-1}{2}
\]
\end{eulerformula}
\begin{eulercomment}
Untuk menyelesaikan sistem persamaan, gunakan vektor persamaan.
Hasilnya adalah vektor solusi.
\end{eulercomment}
\begin{eulerprompt}
>sol &= solve([x+y=3,x^2+y^2=5],[x,y]); $&sol, $&x*y with sol[1]
\end{eulerprompt}
\begin{eulerformula}
\[
\left[ \left[ x=2 , y=1 \right]  , \left[ x=1 , y=2 \right] 
  \right] 
\]
\end{eulerformula}
\begin{eulerformula}
\[
2
\]
\end{eulerformula}
\begin{eulercomment}
Ekspresi simbolik dapat memiliki tanda, yang menunjukkan perlakuan
khusus di Maxima. Beberapa tanda dapat digunakan sebagai perintah
juga, yang lainnya tidak. Tanda ditambahkan dengan "\textbar{}" (bentuk yang
lebih baik dari "ev(...,flags)")
\end{eulercomment}
\begin{eulerprompt}
>$& diff((x^3-1)/(x+1),x) //turunan bentuk pecahan
\end{eulerprompt}
\begin{eulerformula}
\[
\frac{3\,x^2}{x+1}-\frac{x^3-1}{\left(x+1\right)^2}
\]
\end{eulerformula}
\begin{eulerprompt}
>$& diff((x^3-1)/(x+1),x) | ratsimp //menyederhanakan pecahan
\end{eulerprompt}
\begin{eulerformula}
\[
\frac{2\,x^3+3\,x^2+1}{x^2+2\,x+1}
\]
\end{eulerformula}
\begin{eulerprompt}
>$&factor(%)
\end{eulerprompt}
\begin{eulerformula}
\[
\frac{2\,x^3+3\,x^2+1}{\left(x+1\right)^2}
\]
\end{eulerformula}
\eulerheading{Fungsi}
\begin{eulercomment}
Dalam EMT, fungsi adalah program yang didefinisikan dengan perintah
"function". Fungsi ini dapat berupa fungsi satu baris atau fungsi
multibaris.\\
Fungsi satu baris dapat berupa numerik atau simbolik. Fungsi satu
baris numerik didefinisikan oleh ":=".
\end{eulercomment}
\begin{eulerprompt}
>function f(x) := x*sqrt(x^2+1)
\end{eulerprompt}
\begin{eulercomment}
Sebagai gambaran umum, kami tampilkan semua definisi yang mungkin
untuk fungsi satu baris. Suatu fungsi dapat dievaluasi seperti fungsi
Euler bawaan lainnya.
\end{eulercomment}
\begin{eulerprompt}
>f(2)
\end{eulerprompt}
\begin{euleroutput}
  4.472135955
\end{euleroutput}
\begin{eulercomment}
Fungsi ini juga akan bekerja untuk vektor, mematuhi bahasa matriks
Euler, karena ekspresi yang digunakan dalam fungsi tersebut
divektorkan.
\end{eulercomment}
\begin{eulerprompt}
>f(0:0.1:1)
\end{eulerprompt}
\begin{euleroutput}
  [0,  0.100499,  0.203961,  0.313209,  0.430813,  0.559017,  0.699714,
  0.854459,  1.0245,  1.21083,  1.41421]
\end{euleroutput}
\begin{eulercomment}
Fungsi dapat diplot. Alih-alih ekspresi, kita hanya perlu memberikan
nama fungsi.

Berbeda dengan ekspresi simbolik atau numerik, nama fungsi harus
diberikan dalam bentuk string.
\end{eulercomment}
\begin{eulerprompt}
>solve("f",1,y=1)
\end{eulerprompt}
\begin{euleroutput}
  0.786151377757
\end{euleroutput}
\begin{eulercomment}
Secara default, jika Anda perlu menimpa fungsi bawaan, Anda harus
menambahkan kata kunci "overwrite". Menimpakan fungsi bawaan berbahaya
dan dapat menyebabkan masalah bagi fungsi lain yang bergantung
padanya.

Anda masih dapat memanggil fungsi bawaan sebagai "\_...", jika itu
adalah fungsi di inti Euler.
\end{eulercomment}
\begin{eulerprompt}
>function overwrite sin (x) := _sin(x°) // redine sine in degrees
>sin(45)
\end{eulerprompt}
\begin{euleroutput}
  0.707106781187
\end{euleroutput}
\begin{eulercomment}
Sebaiknya kita hilangkan pendefinisian ulang sin.
\end{eulercomment}
\begin{eulerprompt}
>forget sin; sin(pi/4)
\end{eulerprompt}
\begin{euleroutput}
  0.707106781187
\end{euleroutput}
\eulersubheading{Parameter Default}
\begin{eulercomment}
Fungsi numerik dapat memiliki parameter default.
\end{eulercomment}
\begin{eulerprompt}
>function f(x,a=1) := a*x^2
\end{eulerprompt}
\begin{eulercomment}
Mengabaikan parameter ini akan menggunakan nilai default.
\end{eulercomment}
\begin{eulerprompt}
>f(4)
\end{eulerprompt}
\begin{euleroutput}
  16
\end{euleroutput}
\begin{eulercomment}
Mengaturnya akan menimpa nilai default.
\end{eulercomment}
\begin{eulerprompt}
>f(4,5)
\end{eulerprompt}
\begin{euleroutput}
  80
\end{euleroutput}
\begin{eulercomment}
Parameter yang ditetapkan juga akan menimpanya. Ini digunakan oleh
banyak fungsi Euler seperti plot2d, plot3d.
\end{eulercomment}
\begin{eulerprompt}
>f(4,a=1)
\end{eulerprompt}
\begin{euleroutput}
  16
\end{euleroutput}
\begin{eulercomment}
Jika suatu variabel bukan parameter, maka variabel tersebut harus
bersifat global. Fungsi satu baris dapat melihat variabel global.
\end{eulercomment}
\begin{eulerprompt}
>function f(x) := a*x^2
>a=6; f(2)
\end{eulerprompt}
\begin{euleroutput}
  24
\end{euleroutput}
\begin{eulercomment}
Namun, parameter yang ditetapkan akan menggantikan nilai global.

Jika argumen tidak ada dalam daftar parameter yang telah ditetapkan
sebelumnya, argumen tersebut harus dideklarasikan dengan ":="!
\end{eulercomment}
\begin{eulerprompt}
>f(2,a:=5)
\end{eulerprompt}
\begin{euleroutput}
  20
\end{euleroutput}
\begin{eulercomment}
Fungsi simbolik didefinisikan dengan "\&=". Fungsi ini didefinisikan
dalam Euler dan Maxima, dan berfungsi di kedua dunia. Ekspresi yang
mendefinisikan dijalankan melalui Maxima sebelum definisi.
\end{eulercomment}
\begin{eulerprompt}
>function g(x) &= x^3-x*exp(-x); $&g(x)
\end{eulerprompt}
\begin{eulerformula}
\[
x^3-x\,e^ {- x }
\]
\end{eulerformula}
\begin{eulercomment}
Fungsi simbolik dapat digunakan dalam ekspresi simbolik.
\end{eulercomment}
\begin{eulerprompt}
>$&diff(g(x),x), $&% with x=4/3
\end{eulerprompt}
\begin{eulerformula}
\[
x\,e^ {- x }-e^ {- x }+3\,x^2
\]
\end{eulerformula}
\begin{eulerformula}
\[
\frac{e^ {- \frac{4}{3} }}{3}+\frac{16}{3}
\]
\end{eulerformula}
\begin{eulercomment}
Mereka juga dapat digunakan dalam ekspresi numerik. Tentu saja, ini
hanya akan berfungsi jika EMT dapat menginterpretasikan semua hal di
dalam fungsi tersebut.
\end{eulercomment}
\begin{eulerprompt}
>g(5+g(1))
\end{eulerprompt}
\begin{euleroutput}
  178.635099908
\end{euleroutput}
\begin{eulercomment}
Mereka dapat digunakan untuk mendefinisikan fungsi atau ekspresi
simbolis lainnya.
\end{eulercomment}
\begin{eulerprompt}
>function G(x) &= factor(integrate(g(x),x)); $&G(c) // integrate: mengintegralkan
\end{eulerprompt}
\begin{eulerformula}
\[
\frac{e^ {- c }\,\left(c^4\,e^{c}+4\,c+4\right)}{4}
\]
\end{eulerformula}
\begin{eulerprompt}
>solve(&g(x),0.5)
\end{eulerprompt}
\begin{euleroutput}
  0.703467422498
\end{euleroutput}
\begin{eulercomment}
Berikut ini juga berfungsi, karena Euler menggunakan ekspresi simbolik
dalam fungsi g, jika tidak menemukan variabel simbolik g, dan jika ada
fungsi simbolik g.
\end{eulercomment}
\begin{eulerprompt}
>solve(&g,0.5)
\end{eulerprompt}
\begin{euleroutput}
  0.703467422498
\end{euleroutput}
\begin{eulerprompt}
>function P(x,n) &= (2*x-1)^n; $&P(x,n)
\end{eulerprompt}
\begin{eulerformula}
\[
\left(2\,x-1\right)^{n}
\]
\end{eulerformula}
\begin{eulerprompt}
>function Q(x,n) &= (x+2)^n; $&Q(x,n)
\end{eulerprompt}
\begin{eulerformula}
\[
\left(x+2\right)^{n}
\]
\end{eulerformula}
\begin{eulerprompt}
>$&P(x,4), $&expand(%)
\end{eulerprompt}
\begin{eulerformula}
\[
\left(2\,x-1\right)^4
\]
\end{eulerformula}
\begin{eulerformula}
\[
16\,x^4-32\,x^3+24\,x^2-8\,x+1
\]
\end{eulerformula}
\begin{eulerprompt}
>P(3,4)
\end{eulerprompt}
\begin{euleroutput}
  625
\end{euleroutput}
\begin{eulerprompt}
>$&P(x,4)+ Q(x,3), $&expand(%)
\end{eulerprompt}
\begin{eulerformula}
\[
\left(2\,x-1\right)^4+\left(x+2\right)^3
\]
\end{eulerformula}
\begin{eulerformula}
\[
16\,x^4-31\,x^3+30\,x^2+4\,x+9
\]
\end{eulerformula}
\begin{eulerprompt}
>$&P(x,4)-Q(x,3), $&expand(%), $&factor(%)
\end{eulerprompt}
\begin{eulerformula}
\[
\left(2\,x-1\right)^4-\left(x+2\right)^3
\]
\end{eulerformula}
\begin{eulerformula}
\[
16\,x^4-33\,x^3+18\,x^2-20\,x-7
\]
\end{eulerformula}
\begin{eulerformula}
\[
16\,x^4-33\,x^3+18\,x^2-20\,x-7
\]
\end{eulerformula}
\begin{eulerprompt}
>$&P(x,4)*Q(x,3), $&expand(%), $&factor(%)
\end{eulerprompt}
\begin{eulerformula}
\[
\left(x+2\right)^3\,\left(2\,x-1\right)^4
\]
\end{eulerformula}
\begin{eulerformula}
\[
16\,x^7+64\,x^6+24\,x^5-120\,x^4-15\,x^3+102\,x^2-52\,x+8
\]
\end{eulerformula}
\begin{eulerformula}
\[
\left(x+2\right)^3\,\left(2\,x-1\right)^4
\]
\end{eulerformula}
\begin{eulerprompt}
>$&P(x,4)/Q(x,1), $&expand(%), $&factor(%)
\end{eulerprompt}
\begin{eulerformula}
\[
\frac{\left(2\,x-1\right)^4}{x+2}
\]
\end{eulerformula}
\begin{eulerformula}
\[
\frac{16\,x^4}{x+2}-\frac{32\,x^3}{x+2}+\frac{24\,x^2}{x+2}-\frac{8
 \,x}{x+2}+\frac{1}{x+2}
\]
\end{eulerformula}
\begin{eulerformula}
\[
\frac{\left(2\,x-1\right)^4}{x+2}
\]
\end{eulerformula}
\begin{eulerprompt}
>function f(x) &= x^3-x; $&f(x)
\end{eulerprompt}
\begin{eulerformula}
\[
x^3-x
\]
\end{eulerformula}
\begin{eulercomment}
Dengan \&= fungsinya bersifat simbolis, dan dapat digunakan dalam
ekspresi simbolis lainnya.
\end{eulercomment}
\begin{eulerprompt}
>$&integrate(f(x),x)
\end{eulerprompt}
\begin{eulerformula}
\[
\frac{x^4}{4}-\frac{x^2}{2}
\]
\end{eulerformula}
\begin{eulercomment}
Dengan := fungsinya bersifat numerik. Contoh yang bagus adalah
integral tentu seperti

\end{eulercomment}
\begin{eulerformula}
\[
f(x) = \int_1^x t^t \, dt,
\]
\end{eulerformula}
\begin{eulercomment}
yang tidak dapat dievaluasi secara simbolis.

Jika kita mendefinisikan ulang fungsi tersebut dengan kata kunci
"map", fungsi tersebut dapat digunakan untuk vektor x. Secara
internal, fungsi tersebut dipanggil untuk semua nilai x satu kali, dan
hasilnya disimpan dalam vektor.
\end{eulercomment}
\begin{eulerprompt}
>function map f(x) := integrate("x^x",1,x)
>f(0:0.5:2)
\end{eulerprompt}
\begin{euleroutput}
  [-0.783431,  -0.410816,  0,  0.676863,  2.05045]
\end{euleroutput}
\begin{eulercomment}
Fungsi dapat memiliki nilai default untuk parameter.
\end{eulercomment}
\begin{eulerprompt}
>function mylog (x,base=10) := ln(x)/ln(base);
\end{eulerprompt}
\begin{eulercomment}
Sekarang fungsi tersebut dapat dipanggil dengan atau tanpa parameter
"base".
\end{eulercomment}
\begin{eulerprompt}
>mylog(100), mylog(2^6.7,2)
\end{eulerprompt}
\begin{euleroutput}
  2
  6.7
\end{euleroutput}
\begin{eulercomment}
Selain itu, dimungkinkan untuk menggunakan parameter yang ditetapkan.
\end{eulercomment}
\begin{eulerprompt}
>mylog(E^2,base=E)
\end{eulerprompt}
\begin{euleroutput}
  2
\end{euleroutput}
\begin{eulercomment}
Sering kali, kita ingin menggunakan fungsi untuk vektor di satu
tempat, dan untuk elemen individual di tempat lain. Hal ini
dimungkinkan dengan parameter vektor.
\end{eulercomment}
\begin{eulerprompt}
>function f([a,b]) &= a^2+b^2-a*b+b; $&f(a,b), $&f(x,y)
\end{eulerprompt}
\begin{eulerformula}
\[
b^2-a\,b+b+a^2
\]
\end{eulerformula}
\begin{eulerformula}
\[
y^2-x\,y+y+x^2
\]
\end{eulerformula}
\begin{eulercomment}
Fungsi simbolik semacam itu dapat digunakan untuk variabel simbolik.

Namun, fungsi tersebut juga dapat digunakan untuk vektor numerik.
\end{eulercomment}
\begin{eulerprompt}
>v=[3,4]; f(v)
\end{eulerprompt}
\begin{euleroutput}
  17
\end{euleroutput}
\begin{eulercomment}
Ada pula fungsi yang murni simbolis, yang tidak dapat digunakan secara
numerik.
\end{eulercomment}
\begin{eulerprompt}
>function lapl(expr,x,y) &&= diff(expr,x,2)+diff(expr,y,2)//turunan parsial kedua
\end{eulerprompt}
\begin{euleroutput}
  
                   diff(expr, y, 2) + diff(expr, x, 2)
  
\end{euleroutput}
\begin{eulerprompt}
>$&realpart((x+I*y)^4), $&lapl(%,x,y)
\end{eulerprompt}
\begin{eulerformula}
\[
y^4-6\,x^2\,y^2+x^4
\]
\end{eulerformula}
\begin{eulerformula}
\[
0
\]
\end{eulerformula}
\begin{eulercomment}
Namun tentu saja, mereka dapat digunakan dalam ekspresi simbolik atau
dalam definisi fungsi simbolik.
\end{eulercomment}
\begin{eulerprompt}
>function f(x,y) &= factor(lapl((x+y^2)^5,x,y)); $&f(x,y)
\end{eulerprompt}
\begin{eulerformula}
\[
10\,\left(y^2+x\right)^3\,\left(9\,y^2+x+2\right)
\]
\end{eulerformula}
\begin{eulercomment}
Singkatnya

- \&= mendefinisikan fungsi simbolik,\\
- := mendefinisikan fungsi numerik,\\
- \&\&= mendefinisikan fungsi simbolik murni.

\begin{eulercomment}
\eulerheading{Menyelesaikan Ekspresi}
\begin{eulercomment}
Ekspresi dapat diselesaikan secara numerik dan simbolik.

Untuk menyelesaikan ekspresi sederhana dari satu variabel, kita dapat
menggunakan fungsi solve(). Fungsi ini memerlukan nilai awal untuk
memulai pencarian. Secara internal, solve() menggunakan metode secant.
\end{eulercomment}
\begin{eulerprompt}
>solve("x^2-2",1)
\end{eulerprompt}
\begin{euleroutput}
  1.41421356237
\end{euleroutput}
\begin{eulercomment}
Ini juga berlaku untuk ekspresi simbolik. Ambil fungsi berikut.
\end{eulercomment}
\begin{eulerprompt}
>$&solve(x^2=2,x)
\end{eulerprompt}
\begin{eulerformula}
\[
\left[ x=-\sqrt{2} , x=\sqrt{2} \right] 
\]
\end{eulerformula}
\begin{eulerprompt}
>$&solve(x^2-2,x)
\end{eulerprompt}
\begin{eulerformula}
\[
\left[ x=-\sqrt{2} , x=\sqrt{2} \right] 
\]
\end{eulerformula}
\begin{eulerprompt}
>$&solve(a*x^2+b*x+c=0,x)
\end{eulerprompt}
\begin{eulerformula}
\[
\left[ x=\frac{-\sqrt{b^2-4\,a\,c}-b}{2\,a} , x=\frac{\sqrt{b^2-4\,
 a\,c}-b}{2\,a} \right] 
\]
\end{eulerformula}
\begin{eulerprompt}
>$&solve([a*x+b*y=c,d*x+e*y=f],[x,y])
\end{eulerprompt}
\begin{eulerformula}
\[
\left[ \left[ x=-\frac{c\,e}{b\,\left(d-5\right)-a\,e} , y=\frac{c
 \,\left(d-5\right)}{b\,\left(d-5\right)-a\,e} \right]  \right] 
\]
\end{eulerformula}
\begin{eulerprompt}
>px &= 4*x^8+x^7-x^4-x; $&px
\end{eulerprompt}
\begin{eulerformula}
\[
4\,x^8+x^7-x^4-x
\]
\end{eulerformula}
\begin{eulercomment}
Sekarang kita cari titik, di mana polinomialnya adalah 2. Dalam
solve(), nilai target default y=0 dapat diubah dengan variabel yang
ditetapkan.\\
Kita gunakan y=2 dan periksa dengan mengevaluasi polinomial pada hasil
sebelumnya.
\end{eulercomment}
\begin{eulerprompt}
>solve(px,1,y=2), px(%)
\end{eulerprompt}
\begin{euleroutput}
  0.966715594851
  2
\end{euleroutput}
\begin{eulercomment}
Memecahkan ekspresi simbolik dalam bentuk simbolik akan menghasilkan
daftar solusi. Kami menggunakan pemecah simbolik solve() yang
disediakan oleh Maxima.
\end{eulercomment}
\begin{eulerprompt}
>sol &= solve(x^2-x-1,x); $&sol
\end{eulerprompt}
\begin{eulerformula}
\[
\left[ x=\frac{1-\sqrt{5}}{2} , x=\frac{\sqrt{5}+1}{2} \right] 
\]
\end{eulerformula}
\begin{eulercomment}
Cara termudah untuk mendapatkan nilai numerik adalah dengan
mengevaluasi solusi secara numerik seperti sebuah ekspresi.
\end{eulercomment}
\begin{eulerprompt}
>longest sol()
\end{eulerprompt}
\begin{euleroutput}
      -0.6180339887498949       1.618033988749895 
\end{euleroutput}
\begin{eulercomment}
Untuk menggunakan solusi secara simbolis dalam ekspresi lain, cara
termudah adalah "with".
\end{eulercomment}
\begin{eulerprompt}
>$&x^2 with sol[1], $&expand(x^2-x-1 with sol[2])
\end{eulerprompt}
\begin{eulerformula}
\[
\frac{\left(\sqrt{5}-1\right)^2}{4}
\]
\end{eulerformula}
\begin{eulerformula}
\[
0
\]
\end{eulerformula}
\begin{eulercomment}
Memecahkan sistem persamaan secara simbolis dapat dilakukan dengan
vektor persamaan dan penyelesai simbolis solve(). Jawabannya adalah
daftar persamaan.
\end{eulercomment}
\begin{eulerprompt}
>$&solve([x+y=2,x^3+2*y+x=4],[x,y])
\end{eulerprompt}
\begin{eulerformula}
\[
\left[ \left[ x=-1 , y=3 \right]  , \left[ x=1 , y=1 \right]  , 
 \left[ x=0 , y=2 \right]  \right] 
\]
\end{eulerformula}
\begin{eulercomment}
Fungsi f() dapat melihat variabel global. Namun, sering kali kita
ingin menggunakan parameter lokal.

\end{eulercomment}
\begin{eulerformula}
\[
a^x-x^a = 0,1
\]
\end{eulerformula}
\begin{eulercomment}
dengan a=3.
\end{eulercomment}
\begin{eulerprompt}
>function f(x,a) := x^a-a^x;
\end{eulerprompt}
\begin{eulercomment}
Salah satu cara untuk meneruskan parameter tambahan ke f() adalah
dengan menggunakan daftar dengan nama fungsi dan parameter (cara
lainnya adalah parameter titik koma).
\end{eulercomment}
\begin{eulerprompt}
>solve(\{\{"f",3\}\},2,y=0.1)
\end{eulerprompt}
\begin{euleroutput}
  2.54116291558
\end{euleroutput}
\begin{eulercomment}
Ini juga berlaku untuk ekspresi. Namun, elemen daftar bernama harus
digunakan. (Informasi lebih lanjut tentang daftar ada di tutorial
tentang sintaks EMT).
\end{eulercomment}
\begin{eulerprompt}
>solve(\{\{"x^a-a^x",a=3\}\},2,y=0.1)
\end{eulerprompt}
\begin{euleroutput}
  2.54116291558
\end{euleroutput}
\eulerheading{Menyelesaikan Pertidaksamaan}
\begin{eulercomment}
Untuk menyelesaikan pertidaksamaan, EMT tidak akan dapat melakukannya,
melainkan dengan bantuan Maxima, artinya secara eksak (simbolik).
Perintah Maxima yang digunakan adalah fourier\_elim(), yang harus
dipanggil dengan perintah "load(fourier\_elim)" terlebih dahulu.
\end{eulercomment}
\begin{eulerprompt}
>&load(fourier_elim)
\end{eulerprompt}
\begin{euleroutput}
  
          D:/Euler x64/maxima/share/maxima/5.35.1/share/fourier_elim/fo\(\backslash\)
  urier_elim.lisp
  
\end{euleroutput}
\begin{eulerprompt}
>$&fourier_elim([x^2 - 1>0],[x]) // x^2-1 > 0
\end{eulerprompt}
\begin{eulerformula}
\[
\left[ 1<x \right] \lor \left[ x<-1 \right] 
\]
\end{eulerformula}
\begin{eulerprompt}
>$&fourier_elim([x^2 - 1<0],[x]) // x^2-1 < 0
\end{eulerprompt}
\begin{eulerformula}
\[
\left[ -1<x , x<1 \right] 
\]
\end{eulerformula}
\begin{eulerprompt}
>$&fourier_elim([x^2 - 1 # 0],[x]) // x^-1 <> 0
\end{eulerprompt}
\begin{eulerformula}
\[
\left[ -1<x , x<1 \right] \lor \left[ 1<x \right] \lor \left[ x<-1
  \right] 
\]
\end{eulerformula}
\begin{eulerprompt}
>$&fourier_elim([x # 6],[x])
\end{eulerprompt}
\begin{eulerformula}
\[
\left[ x<6 \right] \lor \left[ 6<x \right] 
\]
\end{eulerformula}
\begin{eulerprompt}
>$&fourier_elim([x < 1, x > 1],[x]) // tidak memiliki penyelesaian
\end{eulerprompt}
\begin{eulerformula}
\[
{\it emptyset}
\]
\end{eulerformula}
\begin{eulerprompt}
>$&fourier_elim([minf < x, x < inf],[x]) // solusinya R
\end{eulerprompt}
\begin{eulerformula}
\[
{\it universalset}
\]
\end{eulerformula}
\begin{eulerprompt}
>$&fourier_elim([x^3 - 1 > 0],[x])
\end{eulerprompt}
\begin{eulerformula}
\[
\left[ 1<x , x^2+x+1>0 \right] \lor \left[ x<1 , -x^2-x-1>0
  \right] 
\]
\end{eulerformula}
\begin{eulerprompt}
>$&fourier_elim([cos(x) < 1/2],[x]) // ??? gagal
\end{eulerprompt}
\begin{eulerformula}
\[
\left[ 1-2\,\cos x>0 \right] 
\]
\end{eulerformula}
\begin{eulerprompt}
>$&fourier_elim([y-x < 5, x - y < 7, 10 < y],[x,y]) // sistem pertidaksamaan
\end{eulerprompt}
\begin{eulerformula}
\[
\left[ y-5<x , x<y+7 , 10<y \right] 
\]
\end{eulerformula}
\begin{eulerprompt}
>$&fourier_elim([y-x < 5, x - y < 7, 10 < y],[y,x])
\end{eulerprompt}
\begin{eulerformula}
\[
\left[ {\it max}\left(10 , x-7\right)<y , y<x+5 , 5<x \right] 
\]
\end{eulerformula}
\begin{eulerprompt}
>$&fourier_elim((x + y < 5) and (x - y >8),[x,y])
\end{eulerprompt}
\begin{eulerformula}
\[
\left[ y+8<x , x<5-y , y<-\frac{3}{2} \right] 
\]
\end{eulerformula}
\begin{eulerprompt}
>$&fourier_elim(((x + y < 5) and x < 1) or  (x - y >8),[x,y])
\end{eulerprompt}
\begin{eulerformula}
\[
\left[ y+8<x \right] \lor \left[ x<{\it min}\left(1 , 5-y\right)
  \right] 
\]
\end{eulerformula}
\begin{eulerprompt}
>&fourier_elim([max(x,y) > 6, x # 8, abs(y-1) > 12],[x,y])
\end{eulerprompt}
\begin{euleroutput}
  
          [6 < x, x < 8, y < - 11] or [8 < x, y < - 11]
   or [x < 8, 13 < y] or [x = y, 13 < y] or [8 < x, x < y, 13 < y]
   or [y < x, 13 < y]
  
\end{euleroutput}
\begin{eulerprompt}
>$&fourier_elim([(x+6)/(x-9) <= 6],[x])
\end{eulerprompt}
\begin{eulerformula}
\[
\left[ x=12 \right] \lor \left[ 12<x \right] \lor \left[ x<9
  \right] 
\]
\end{eulerformula}
\eulerheading{Bahasa Matriks}
\begin{eulercomment}
Dokumentasi inti EMT berisi pembahasan terperinci tentang bahasa
matriks Euler.

Vektor dan matriks dimasukkan dengan tanda kurung siku, elemen
dipisahkan dengan koma, baris dipisahkan dengan titik koma.
\end{eulercomment}
\begin{eulerprompt}
>A=[1,2;3,4]
\end{eulerprompt}
\begin{euleroutput}
              1             2 
              3             4 
\end{euleroutput}
\begin{eulercomment}
The matrix product is denoted by a dot.
\end{eulercomment}
\begin{eulerprompt}
>b=[3;4]
\end{eulerprompt}
\begin{euleroutput}
              3 
              4 
\end{euleroutput}
\begin{eulerprompt}
>b' // transpose b
\end{eulerprompt}
\begin{euleroutput}
  [3,  4]
\end{euleroutput}
\begin{eulerprompt}
>inv(A) //inverse A
\end{eulerprompt}
\begin{euleroutput}
             -2             1 
            1.5          -0.5 
\end{euleroutput}
\begin{eulerprompt}
>A.b //perkalian matriks
\end{eulerprompt}
\begin{euleroutput}
             11 
             25 
\end{euleroutput}
\begin{eulerprompt}
>A.inv(A)
\end{eulerprompt}
\begin{euleroutput}
              1             0 
              0             1 
\end{euleroutput}
\begin{eulercomment}
Poin utama dari bahasa matriks adalah bahwa semua fungsi dan operator
bekerja elemen demi elemen.
\end{eulercomment}
\begin{eulerprompt}
>A.A
\end{eulerprompt}
\begin{euleroutput}
              7            10 
             15            22 
\end{euleroutput}
\begin{eulerprompt}
>A^2 //perpangkatan elemen2 A
\end{eulerprompt}
\begin{euleroutput}
              1             4 
              9            16 
\end{euleroutput}
\begin{eulerprompt}
>A.A.A
\end{eulerprompt}
\begin{euleroutput}
             37            54 
             81           118 
\end{euleroutput}
\begin{eulerprompt}
>power(A,3) //perpangkatan matriks
\end{eulerprompt}
\begin{euleroutput}
             37            54 
             81           118 
\end{euleroutput}
\begin{eulerprompt}
>A/A //pembagian elemen-elemen matriks yang seletak
\end{eulerprompt}
\begin{euleroutput}
              1             1 
              1             1 
\end{euleroutput}
\begin{eulerprompt}
>A/b //pembagian elemen2 A oleh elemen2 b kolom demi kolom (karena b vektor kolom)
\end{eulerprompt}
\begin{euleroutput}
       0.333333      0.666667 
           0.75             1 
\end{euleroutput}
\begin{eulerprompt}
>A\(\backslash\)b // hasilkali invers A dan b, A^(-1)b 
\end{eulerprompt}
\begin{euleroutput}
             -2 
            2.5 
\end{euleroutput}
\begin{eulerprompt}
>inv(A).b
\end{eulerprompt}
\begin{euleroutput}
             -2 
            2.5 
\end{euleroutput}
\begin{eulerprompt}
>A\(\backslash\)A   //A^(-1)A
\end{eulerprompt}
\begin{euleroutput}
              1             0 
              0             1 
\end{euleroutput}
\begin{eulerprompt}
>inv(A).A
\end{eulerprompt}
\begin{euleroutput}
              1             0 
              0             1 
\end{euleroutput}
\begin{eulerprompt}
>A*A //perkalin elemen-elemen matriks seletak
\end{eulerprompt}
\begin{euleroutput}
              1             4 
              9            16 
\end{euleroutput}
\begin{eulercomment}
Ini bukan hasil perkalian matriks, tetapi perkalian elemen demi
elemen. Hal yang sama berlaku untuk vektor.
\end{eulercomment}
\begin{eulerprompt}
>b^2 // perpangkatan elemen-elemen matriks/vektor
\end{eulerprompt}
\begin{euleroutput}
              9 
             16 
\end{euleroutput}
\begin{eulercomment}
Jika salah satu operan merupakan vektor atau skalar, ia diekspansi
dengan cara alami.
\end{eulercomment}
\begin{eulerprompt}
>2*A
\end{eulerprompt}
\begin{euleroutput}
              2             4 
              6             8 
\end{euleroutput}
\begin{eulercomment}
Misalnya, jika operan adalah vektor kolom, elemen-elemennya diterapkan
ke semua baris A.
\end{eulercomment}
\begin{eulerprompt}
>[1,2]*A
\end{eulerprompt}
\begin{euleroutput}
              1             4 
              3             8 
\end{euleroutput}
\begin{eulercomment}
Jika itu adalah vektor baris maka diterapkan ke semua kolom A.
\end{eulercomment}
\begin{eulerprompt}
>A*[2,3]
\end{eulerprompt}
\begin{euleroutput}
              2             6 
              6            12 
\end{euleroutput}
\begin{eulercomment}
Seseorang dapat membayangkan perkalian ini seolah-olah vektor baris v
telah diduplikasi untuk membentuk matriks berukuran sama dengan A.
\end{eulercomment}
\begin{eulerprompt}
>dup([1,2],2) // dup: menduplikasi/menggandakan vektor [1,2] sebanyak 2 kali (baris)
\end{eulerprompt}
\begin{euleroutput}
              1             2 
              1             2 
\end{euleroutput}
\begin{eulerprompt}
>A*dup([1,2],2) 
\end{eulerprompt}
\begin{euleroutput}
              1             4 
              3             8 
\end{euleroutput}
\begin{eulercomment}
Hal ini juga berlaku untuk dua vektor, yang satu merupakan vektor
baris dan yang lainnya merupakan vektor kolom. Kita menghitung i*j
untuk i,j dari 1 hingga 5. Caranya adalah dengan mengalikan 1:5 dengan
transposenya. Bahasa matriks Euler secara otomatis menghasilkan tabel
nilai.
\end{eulercomment}
\begin{eulerprompt}
>(1:5)*(1:5)' // hasilkali elemen-elemen vektor baris dan vektor kolom
\end{eulerprompt}
\begin{euleroutput}
              1             2             3             4             5 
              2             4             6             8            10 
              3             6             9            12            15 
              4             8            12            16            20 
              5            10            15            20            25 
\end{euleroutput}
\begin{eulercomment}
Sekali lagi, ingatlah bahwa ini bukan produk matriks!
\end{eulercomment}
\begin{eulerprompt}
>(1:5).(1:5)' // hasilkali vektor baris dan vektor kolom
\end{eulerprompt}
\begin{euleroutput}
  55
\end{euleroutput}
\begin{eulerprompt}
>sum((1:5)*(1:5)) // sama hasilnya
\end{eulerprompt}
\begin{euleroutput}
  55
\end{euleroutput}
\begin{eulercomment}
Bahkan operator seperti \textless{} atau == bekerja dengan cara yang sama.
\end{eulercomment}
\begin{eulerprompt}
>(1:10)<6 // menguji elemen-elemen yang kurang dari 6
\end{eulerprompt}
\begin{euleroutput}
  [1,  1,  1,  1,  1,  0,  0,  0,  0,  0]
\end{euleroutput}
\begin{eulercomment}
Misalnya, kita dapat menghitung jumlah elemen yang memenuhi kondisi
tertentu dengan fungsi sum().
\end{eulercomment}
\begin{eulerprompt}
>sum((1:10)<6) // banyak elemen yang kurang dari 6
\end{eulerprompt}
\begin{euleroutput}
  5
\end{euleroutput}
\begin{eulercomment}
Euler memiliki operator perbandingan, seperti "==", yang memeriksa
kesetaraan.

Kita memperoleh vektor 0 dan 1, di mana 1 berarti benar.
\end{eulercomment}
\begin{eulerprompt}
>t=(1:10)^2; t==25 //menguji elemen2 t yang sama dengan 25 (hanya ada 1)
\end{eulerprompt}
\begin{euleroutput}
  [0,  0,  0,  0,  1,  0,  0,  0,  0,  0]
\end{euleroutput}
\begin{eulercomment}
Dari vektor tersebut, "nonzeros" memilih elemen yang bukan nol.

Dalam kasus ini, kita memperoleh indeks semua elemen yang lebih besar
dari 50.
\end{eulercomment}
\begin{eulerprompt}
>nonzeros(t>50) //indeks elemen2 t yang lebih besar daripada 50
\end{eulerprompt}
\begin{euleroutput}
  [8,  9,  10]
\end{euleroutput}
\begin{eulercomment}
Tentu saja, kita dapat menggunakan vektor indeks ini untuk mendapatkan
nilai yang sesuai dalam t.
\end{eulercomment}
\begin{eulerprompt}
>t[nonzeros(t>50)] //elemen2 t yang lebih besar daripada 50
\end{eulerprompt}
\begin{euleroutput}
  [64,  81,  100]
\end{euleroutput}
\begin{eulercomment}
Sebagai contoh, mari kita cari semua kuadrat angka 1 hingga 1000,
yaitu 5 modulo 11 dan 3 modulo 13.
\end{eulercomment}
\begin{eulerprompt}
>t=1:1000; nonzeros(mod(t^2,11)==5 && mod(t^2,13)==3)
\end{eulerprompt}
\begin{euleroutput}
  [4,  48,  95,  139,  147,  191,  238,  282,  290,  334,  381,  425,
  433,  477,  524,  568,  576,  620,  667,  711,  719,  763,  810,  854,
  862,  906,  953,  997]
\end{euleroutput}
\begin{eulercomment}
EMT tidak sepenuhnya efektif untuk komputasi integer. Ia menggunakan
floating point presisi ganda secara internal. Namun, ia sering kali
sangat berguna.

Kita dapat memeriksa keutamaan. Mari kita cari tahu, berapa banyak
kuadrat ditambah 1 yang merupakan bilangan prima.
\end{eulercomment}
\begin{eulerprompt}
>t=1:1000; length(nonzeros(isprime(t^2+1)))
\end{eulerprompt}
\begin{euleroutput}
  112
\end{euleroutput}
\begin{eulercomment}
Fungsi nonzeros() hanya berfungsi untuk vektor. Untuk matriks, ada
mnonzeros().
\end{eulercomment}
\begin{eulerprompt}
>seed(2); A=random(3,4)
\end{eulerprompt}
\begin{euleroutput}
       0.765761      0.401188      0.406347      0.267829 
        0.13673      0.390567      0.495975      0.952814 
       0.548138      0.006085      0.444255      0.539246 
\end{euleroutput}
\begin{eulercomment}
Mengembalikan indeks elemen, yang bukan nol.
\end{eulercomment}
\begin{eulerprompt}
>k=mnonzeros(A<0.4) //indeks elemen2 A yang kurang dari 0,4
\end{eulerprompt}
\begin{euleroutput}
              1             4 
              2             1 
              2             2 
              3             2 
\end{euleroutput}
\begin{eulercomment}
Indeks ini dapat digunakan untuk menetapkan elemen pada nilai
tertentu.
\end{eulercomment}
\begin{eulerprompt}
>mset(A,k,0) //mengganti elemen2 suatu matriks pada indeks tertentu
\end{eulerprompt}
\begin{euleroutput}
       0.765761      0.401188      0.406347             0 
              0             0      0.495975      0.952814 
       0.548138             0      0.444255      0.539246 
\end{euleroutput}
\begin{eulercomment}
Fungsi mset() juga dapat mengatur elemen pada indeks ke entri matriks
lainnya.
\end{eulercomment}
\begin{eulerprompt}
>mset(A,k,-random(size(A)))
\end{eulerprompt}
\begin{euleroutput}
       0.765761      0.401188      0.406347     -0.126917 
      -0.122404     -0.691673      0.495975      0.952814 
       0.548138     -0.483902      0.444255      0.539246 
\end{euleroutput}
\begin{eulercomment}
Dan adalah mungkin untuk mendapatkan unsur-unsur dalam sebuah vektor.
\end{eulercomment}
\begin{eulerprompt}
>mget(A,k)
\end{eulerprompt}
\begin{euleroutput}
  [0.267829,  0.13673,  0.390567,  0.006085]
\end{euleroutput}
\begin{eulercomment}
Fungsi lain yang berguna adalah extrema, yang mengembalikan nilai
minimal dan maksimal di setiap baris matriks dan posisinya.
\end{eulercomment}
\begin{eulerprompt}
>ex=extrema(A)
\end{eulerprompt}
\begin{euleroutput}
       0.267829             4      0.765761             1 
        0.13673             1      0.952814             4 
       0.006085             2      0.548138             1 
\end{euleroutput}
\begin{eulercomment}
Kita dapat menggunakan ini untuk mengekstrak nilai maksimal pada
setiap baris.
\end{eulercomment}
\begin{eulerprompt}
>ex[,3]'
\end{eulerprompt}
\begin{euleroutput}
  [0.765761,  0.952814,  0.548138]
\end{euleroutput}
\begin{eulercomment}
Ini tentu saja sama dengan fungsi max().
\end{eulercomment}
\begin{eulerprompt}
>max(A)'
\end{eulerprompt}
\begin{euleroutput}
  [0.765761,  0.952814,  0.548138]
\end{euleroutput}
\begin{eulercomment}
Tetapi dengan mget(), kita dapat mengekstrak indeks dan menggunakan
informasi ini untuk mengekstrak elemen pada posisi yang sama dari
matriks lain.
\end{eulercomment}
\begin{eulerprompt}
>j=(1:rows(A))'|ex[,4], mget(-A,j)
\end{eulerprompt}
\begin{euleroutput}
              1             1 
              2             4 
              3             1 
  [-0.765761,  -0.952814,  -0.548138]
\end{euleroutput}
\eulerheading{Fungsi Matriks Lainnya (Membangun Matriks)}
\begin{eulercomment}
Untuk membangun sebuah matriks, kita dapat menumpuk satu matriks di
atas matriks lainnya. Jika keduanya tidak memiliki jumlah kolom yang
sama, matriks yang lebih pendek akan diisi dengan 0.
\end{eulercomment}
\begin{eulerprompt}
>v=1:3; v_v
\end{eulerprompt}
\begin{euleroutput}
              1             2             3 
              1             2             3 
\end{euleroutput}
\begin{eulercomment}
Dengan cara yang sama, kita dapat menempelkan suatu matriks ke sisi
lain yang berdampingan, jika keduanya memiliki jumlah baris yang sama.
\end{eulercomment}
\begin{eulerprompt}
>A=random(3,4); A|v'
\end{eulerprompt}
\begin{euleroutput}
       0.032444     0.0534171      0.595713      0.564454             1 
        0.83916      0.175552      0.396988       0.83514             2 
      0.0257573      0.658585      0.629832      0.770895             3 
\end{euleroutput}
\begin{eulercomment}
Jika tidak memiliki jumlah baris yang sama, matriks yang lebih pendek
diisi dengan 0.

Ada pengecualian untuk aturan ini. Bilangan riil yang dilampirkan ke
matriks akan digunakan sebagai kolom yang diisi dengan bilangan riil
tersebut.
\end{eulercomment}
\begin{eulerprompt}
>A|1
\end{eulerprompt}
\begin{euleroutput}
       0.032444     0.0534171      0.595713      0.564454             1 
        0.83916      0.175552      0.396988       0.83514             1 
      0.0257573      0.658585      0.629832      0.770895             1 
\end{euleroutput}
\begin{eulercomment}
Dimungkinkan untuk membuat matriks dari vektor baris dan kolom.
\end{eulercomment}
\begin{eulerprompt}
>[v;v]
\end{eulerprompt}
\begin{euleroutput}
              1             2             3 
              1             2             3 
\end{euleroutput}
\begin{eulerprompt}
>[v',v']
\end{eulerprompt}
\begin{euleroutput}
              1             1 
              2             2 
              3             3 
\end{euleroutput}
\begin{eulercomment}
Tujuan utama dari ini adalah untuk menafsirkan vektor ekspresi untuk
vektor kolom.
\end{eulercomment}
\begin{eulerprompt}
>"[x,x^2]"(v')
\end{eulerprompt}
\begin{euleroutput}
              1             1 
              2             4 
              3             9 
\end{euleroutput}
\begin{eulercomment}
Untuk mendapatkan ukuran A, kita dapat menggunakan fungsi berikut.
\end{eulercomment}
\begin{eulerprompt}
>C=zeros(2,4); rows(C), cols(C), size(C), length(C)
\end{eulerprompt}
\begin{euleroutput}
  2
  4
  [2,  4]
  4
\end{euleroutput}
\begin{eulercomment}
Untuk vektor, ada length().
\end{eulercomment}
\begin{eulerprompt}
>length(2:10)
\end{eulerprompt}
\begin{euleroutput}
  9
\end{euleroutput}
\begin{eulercomment}
Ada banyak fungsi lain yang menghasilkan matriks.
\end{eulercomment}
\begin{eulerprompt}
>ones(2,2)
\end{eulerprompt}
\begin{euleroutput}
              1             1 
              1             1 
\end{euleroutput}
\begin{eulercomment}
Ini juga dapat digunakan dengan satu parameter. Untuk mendapatkan
vektor dengan angka selain 1, gunakan yang berikut ini.
\end{eulercomment}
\begin{eulerprompt}
>ones(5)*6
\end{eulerprompt}
\begin{euleroutput}
  [6,  6,  6,  6,  6]
\end{euleroutput}
\begin{eulercomment}
Matriks bilangan acak juga dapat dihasilkan dengan acak (distribusi
seragam) atau normal (distribusi Gauß).
\end{eulercomment}
\begin{eulerprompt}
>random(2,2)
\end{eulerprompt}
\begin{euleroutput}
        0.66566      0.831835 
          0.977      0.544258 
\end{euleroutput}
\begin{eulercomment}
Berikut adalah fungsi berguna lainnya, yang merestrukturisasi
elemen-elemen suatu matriks menjadi matriks lain.
\end{eulercomment}
\begin{eulerprompt}
>redim(1:9,3,3) // menyusun elemen2 1, 2, 3, ..., 9 ke bentuk matriks 3x3
\end{eulerprompt}
\begin{euleroutput}
              1             2             3 
              4             5             6 
              7             8             9 
\end{euleroutput}
\begin{eulercomment}
Dengan fungsi berikut, kita dapat menggunakan ini dan fungsi dup untuk
menulis fungsi rep(), yang mengulang vektor n kali.
\end{eulercomment}
\begin{eulerprompt}
>function rep(v,n) := redim(dup(v,n),1,n*cols(v))
\end{eulerprompt}
\begin{eulercomment}
Mari kita menguji.
\end{eulercomment}
\begin{eulerprompt}
>rep(1:3,5)
\end{eulerprompt}
\begin{euleroutput}
  [1,  2,  3,  1,  2,  3,  1,  2,  3,  1,  2,  3,  1,  2,  3]
\end{euleroutput}
\begin{eulercomment}
Fungsi multdup() menduplikasi elemen suatu vektor.
\end{eulercomment}
\begin{eulerprompt}
>multdup(1:3,5), multdup(1:3,[2,3,2])
\end{eulerprompt}
\begin{euleroutput}
  [1,  1,  1,  1,  1,  2,  2,  2,  2,  2,  3,  3,  3,  3,  3]
  [1,  1,  2,  2,  2,  3,  3]
\end{euleroutput}
\begin{eulercomment}
Fungsi flipx() dan flipy() membalikkan urutan baris atau kolom
matriks. Yaitu, fungsi flipx() membalik secara horizontal.
\end{eulercomment}
\begin{eulerprompt}
>flipx(1:5) //membalik elemen2 vektor baris
\end{eulerprompt}
\begin{euleroutput}
  [5,  4,  3,  2,  1]
\end{euleroutput}
\begin{eulercomment}
Untuk rotasi, Euler memiliki rotleft() dan rotright().
\end{eulercomment}
\begin{eulerprompt}
>rotleft(1:5) // memutar elemen2 vektor baris
\end{eulerprompt}
\begin{euleroutput}
  [2,  3,  4,  5,  1]
\end{euleroutput}
\begin{eulercomment}
Fungsi khusus adalah drop(v,i), yang menghapus elemen dengan indeks di
i dari vektor v.
\end{eulercomment}
\begin{eulerprompt}
>drop(10:20,3)
\end{eulerprompt}
\begin{euleroutput}
  [10,  11,  13,  14,  15,  16,  17,  18,  19,  20]
\end{euleroutput}
\begin{eulercomment}
Perhatikan bahwa vektor i dalam drop(v,i) merujuk pada indeks elemen
dalam v, bukan nilai elemen. Jika Anda ingin menghapus elemen, Anda
perlu menemukan elemen terlebih dahulu. Fungsi indexof(v,x) dapat
digunakan untuk menemukan elemen x dalam vektor v yang diurutkan.
\end{eulercomment}
\begin{eulerprompt}
>v=primes(50), i=indexof(v,10:20), drop(v,i)
\end{eulerprompt}
\begin{euleroutput}
  [2,  3,  5,  7,  11,  13,  17,  19,  23,  29,  31,  37,  41,  43,  47]
  [0,  5,  0,  6,  0,  0,  0,  7,  0,  8,  0]
  [2,  3,  5,  7,  23,  29,  31,  37,  41,  43,  47]
\end{euleroutput}
\begin{eulercomment}
Seperti yang Anda lihat, tidak ada salahnya menyertakan indeks di luar
rentang (seperti 0), indeks ganda, atau indeks yang tidak diurutkan.
\end{eulercomment}
\begin{eulerprompt}
>drop(1:10,shuffle([0,0,5,5,7,12,12]))
\end{eulerprompt}
\begin{euleroutput}
  [1,  2,  3,  4,  6,  8,  9,  10]
\end{euleroutput}
\begin{eulercomment}
Ada beberapa fungsi khusus untuk mengatur diagonal atau membuat
matriks diagonal.

Kita mulai dengan matriks identitas.
\end{eulercomment}
\begin{eulerprompt}
>A=id(5) // matriks identitas 5x5
\end{eulerprompt}
\begin{euleroutput}
              1             0             0             0             0 
              0             1             0             0             0 
              0             0             1             0             0 
              0             0             0             1             0 
              0             0             0             0             1 
\end{euleroutput}
\begin{eulercomment}
Kemudian kita atur diagonal bawah (-1) menjadi 1:4.
\end{eulercomment}
\begin{eulerprompt}
>setdiag(A,-1,1:4) //mengganti diagonal di bawah diagonal utama
\end{eulerprompt}
\begin{euleroutput}
              1             0             0             0             0 
              1             1             0             0             0 
              0             2             1             0             0 
              0             0             3             1             0 
              0             0             0             4             1 
\end{euleroutput}
\begin{eulercomment}
Perhatikan bahwa kita tidak mengubah matriks A. Kita mendapatkan
matriks baru sebagai hasil dari setdiag().

Berikut ini adalah fungsi yang mengembalikan matriks tri-diagonal.
\end{eulercomment}
\begin{eulerprompt}
>function tridiag (n,a,b,c) := setdiag(setdiag(b*id(n),1,c),-1,a); ...
>tridiag(5,1,2,3)
\end{eulerprompt}
\begin{euleroutput}
              2             3             0             0             0 
              1             2             3             0             0 
              0             1             2             3             0 
              0             0             1             2             3 
              0             0             0             1             2 
\end{euleroutput}
\begin{eulercomment}
Diagonal matriks juga dapat diekstraksi dari matriks. Untuk
menunjukkan hal ini, kami merestrukturisasi vektor 1:9 menjadi matriks
3x3.
\end{eulercomment}
\begin{eulerprompt}
>A=redim(1:9,3,3)
\end{eulerprompt}
\begin{euleroutput}
              1             2             3 
              4             5             6 
              7             8             9 
\end{euleroutput}
\begin{eulercomment}
Sekarang kita dapat mengekstrak diagonalnya.
\end{eulercomment}
\begin{eulerprompt}
>d=getdiag(A,0)
\end{eulerprompt}
\begin{euleroutput}
  [1,  5,  9]
\end{euleroutput}
\begin{eulercomment}
Misalnya, kita dapat membagi matriks berdasarkan diagonalnya. Bahasa
matriks memastikan bahwa vektor kolom d diterapkan ke matriks baris
demi baris.
\end{eulercomment}
\begin{eulerprompt}
>fraction A/d'
\end{eulerprompt}
\begin{euleroutput}
          1         2         3 
        4/5         1       6/5 
        7/9       8/9         1 
\end{euleroutput}
\eulerheading{Vektorisasi}
\begin{eulercomment}
Hampir semua fungsi di Euler juga berfungsi untuk masukan matriks dan
vektor, jika ini masuk akal.

Misalnya, fungsi sqrt() menghitung akar kuadrat dari semua elemen
vektor atau matriks.
\end{eulercomment}
\begin{eulerprompt}
>sqrt(1:3)
\end{eulerprompt}
\begin{euleroutput}
  [1,  1.41421,  1.73205]
\end{euleroutput}
\begin{eulercomment}
Jadi Anda dapat dengan mudah membuat tabel nilai. Ini adalah salah
satu cara untuk memplot fungsi (alternatifnya menggunakan ekspresi).
\end{eulercomment}
\begin{eulerprompt}
>x=1:0.01:5; y=log(x)/x^2; // terlalu panjang untuk ditampikan
\end{eulerprompt}
\begin{eulercomment}
Dengan ini dan operator titik dua a:delta:b, vektor nilai fungsi dapat
dibuat dengan mudah.

Dalam contoh berikut, kita buat vektor nilai t[i] dengan spasi 0,1
dari -1 hingga 1. Kemudian kita buat vektor nilai fungsi

\end{eulercomment}
\begin{eulerformula}
\[
s = t^3-t
\]
\end{eulerformula}
\begin{eulerprompt}
>t=-1:0.1:1; s=t^3-t
\end{eulerprompt}
\begin{euleroutput}
  [0,  0.171,  0.288,  0.357,  0.384,  0.375,  0.336,  0.273,  0.192,
  0.099,  0,  -0.099,  -0.192,  -0.273,  -0.336,  -0.375,  -0.384,
  -0.357,  -0.288,  -0.171,  0]
\end{euleroutput}
\begin{eulercomment}
EMT mengembangkan operator untuk skalar, vektor, dan matriks dengan
cara yang jelas.

Misalnya, vektor kolom dikalikan vektor baris akan mengembang menjadi
matriks, jika operator diterapkan. Berikut ini, v' adalah vektor yang
ditransposisikan (vektor kolom).
\end{eulercomment}
\begin{eulerprompt}
>shortest (1:5)*(1:5)'
\end{eulerprompt}
\begin{euleroutput}
       1      2      3      4      5 
       2      4      6      8     10 
       3      6      9     12     15 
       4      8     12     16     20 
       5     10     15     20     25 
\end{euleroutput}
\begin{eulercomment}
Perhatikan bahwa ini sangat berbeda dari perkalian matriks. Perkalian
matriks dilambangkan dengan titik "." dalam EMT.
\end{eulercomment}
\begin{eulerprompt}
>(1:5).(1:5)'
\end{eulerprompt}
\begin{euleroutput}
  55
\end{euleroutput}
\begin{eulercomment}
Secara default, vektor baris dicetak dalam format ringkas.
\end{eulercomment}
\begin{eulerprompt}
>[1,2,3,4]
\end{eulerprompt}
\begin{euleroutput}
  [1,  2,  3,  4]
\end{euleroutput}
\begin{eulercomment}
Untuk matriks, operator khusus . menunjukkan perkalian matriks, dan A'
menunjukkan transposisi. Matriks 1x1 dapat digunakan seperti bilangan
riil.
\end{eulercomment}
\begin{eulerprompt}
>v:=[1,2]; v.v', %^2
\end{eulerprompt}
\begin{euleroutput}
  5
  25
\end{euleroutput}
\begin{eulercomment}
To transpose a matrix we use the apostrophe.
\end{eulercomment}
\begin{eulerprompt}
>v=1:4; v'
\end{eulerprompt}
\begin{euleroutput}
              1 
              2 
              3 
              4 
\end{euleroutput}
\begin{eulercomment}
So we can compute matrix A times vector b.
\end{eulercomment}
\begin{eulerprompt}
>A=[1,2,3,4;5,6,7,8]; A.v'
\end{eulerprompt}
\begin{euleroutput}
             30 
             70 
\end{euleroutput}
\begin{eulercomment}
Perhatikan bahwa v masih merupakan vektor baris. Jadi v'.v berbeda
dari v.v'.
\end{eulercomment}
\begin{eulerprompt}
>v'.v
\end{eulerprompt}
\begin{euleroutput}
              1             2             3             4 
              2             4             6             8 
              3             6             9            12 
              4             8            12            16 
\end{euleroutput}
\begin{eulercomment}
v.v' menghitung norma v kuadrat untuk vektor baris v. Hasilnya adalah
vektor 1x1, yang bekerja seperti bilangan riil.
\end{eulercomment}
\begin{eulerprompt}
>v.v'
\end{eulerprompt}
\begin{euleroutput}
  30
\end{euleroutput}
\begin{eulercomment}
Ada juga norma fungsi (bersama dengan banyak fungsi Aljabar Linear
lainnya).
\end{eulercomment}
\begin{eulerprompt}
>norm(v)^2
\end{eulerprompt}
\begin{euleroutput}
  30
\end{euleroutput}
\begin{eulercomment}
Operator dan fungsi mematuhi bahasa matriks Euler.

Berikut ini ringkasan aturannya.

- Fungsi yang diterapkan pada vektor atau matriks diterapkan pada
setiap elemen.

- Operator yang beroperasi pada dua matriks dengan ukuran yang sama
diterapkan secara berpasangan pada elemen-elemen matriks.

- Jika kedua matriks memiliki dimensi yang berbeda, keduanya
diekspansi dengan cara yang masuk akal, sehingga memiliki ukuran yang
sama.

Misalnya, nilai skalar dikalikan vektor mengalikan nilai dengan setiap
elemen vektor. Atau matriks dikalikan vektor (dengan *, bukan .)
mengekspansi vektor ke ukuran matriks dengan menduplikasinya.

Berikut ini adalah kasus sederhana dengan operator \textasciicircum{}.
\end{eulercomment}
\begin{eulerprompt}
>[1,2,3]^2
\end{eulerprompt}
\begin{euleroutput}
  [1,  4,  9]
\end{euleroutput}
\begin{eulercomment}
Berikut ini adalah kasus yang lebih rumit. Vektor baris dikalikan
vektor kolom, keduanya diekspansi dengan cara menduplikasi.
\end{eulercomment}
\begin{eulerprompt}
>v:=[1,2,3]; v*v'
\end{eulerprompt}
\begin{euleroutput}
              1             2             3 
              2             4             6 
              3             6             9 
\end{euleroutput}
\begin{eulercomment}
Perhatikan bahwa produk skalar menggunakan produk matriks, bukan *!
\end{eulercomment}
\begin{eulerprompt}
>v.v'
\end{eulerprompt}
\begin{euleroutput}
  14
\end{euleroutput}
\begin{eulercomment}
Ada banyak fungsi untuk matriks. Kami memberikan daftar singkatnya.
Anda harus merujuk ke dokumentasi untuk informasi lebih lanjut tentang
perintah-perintah ini.

sum,prod menghitung jumlah dan hasil perkalian baris-baris

cumsum,cumprod melakukan hal yang sama secara kumulatif

menghitung nilai ekstrem dari setiap baris

extrema mengembalikan vektor dengan informasi ekstrem

diag(A,i) mengembalikan diagonal ke-i

setdiag(A,i,v) menetapkan diagonal ke-i

id(n) matriks identitas

det(A) determinan

charpoly(A) polinomial karakteristik

eigenvalues(A) nilai eigen
\end{eulercomment}
\begin{eulerprompt}
>v*v, sum(v*v), cumsum(v*v)
\end{eulerprompt}
\begin{euleroutput}
  [1,  4,  9]
  14
  [1,  5,  14]
\end{euleroutput}
\begin{eulercomment}
Operator : menghasilkan vektor baris dengan spasi yang sama, secara
opsional dengan ukuran langkah.
\end{eulercomment}
\begin{eulerprompt}
>1:4, 1:2:10
\end{eulerprompt}
\begin{euleroutput}
  [1,  2,  3,  4]
  [1,  3,  5,  7,  9]
\end{euleroutput}
\begin{eulercomment}
Untuk menggabungkan matriks dan vektor ada operator "\textbar{}" dan "\_".
\end{eulercomment}
\begin{eulerprompt}
>[1,2,3]|[4,5], [1,2,3]_1
\end{eulerprompt}
\begin{euleroutput}
  [1,  2,  3,  4,  5]
              1             2             3 
              1             1             1 
\end{euleroutput}
\begin{eulercomment}
Elemen-elemen suatu matriks disebut dengan "A[i,j]".
\end{eulercomment}
\begin{eulerprompt}
>A:=[1,2,3;4,5,6;7,8,9]; A[2,3]
\end{eulerprompt}
\begin{euleroutput}
  6
\end{euleroutput}
\begin{eulercomment}
Untuk vektor baris atau kolom, v[i] adalah elemen ke-i dari vektor.
Untuk matriks, ini mengembalikan baris ke-i lengkap dari matriks.
\end{eulercomment}
\begin{eulerprompt}
>v:=[2,4,6,8]; v[3], A[3]
\end{eulerprompt}
\begin{euleroutput}
  6
  [7,  8,  9]
\end{euleroutput}
\begin{eulercomment}
Indeks juga dapat berupa vektor baris indeks. : menunjukkan semua
indeks.
\end{eulercomment}
\begin{eulerprompt}
>v[1:2], A[:,2]
\end{eulerprompt}
\begin{euleroutput}
  [2,  4]
              2 
              5 
              8 
\end{euleroutput}
\begin{eulercomment}
Bentuk singkat dari : adalah menghilangkan indeks sepenuhnya.
\end{eulercomment}
\begin{eulerprompt}
>A[,2:3]
\end{eulerprompt}
\begin{euleroutput}
              2             3 
              5             6 
              8             9 
\end{euleroutput}
\begin{eulercomment}
Untuk tujuan vektorisasi, elemen-elemen matriks dapat diakses
seolah-olah mereka adalah vektor.
\end{eulercomment}
\begin{eulerprompt}
>A\{4\}
\end{eulerprompt}
\begin{euleroutput}
  4
\end{euleroutput}
\begin{eulercomment}
Matriks juga dapat diratakan, menggunakan fungsi redim(). Hal ini
diimplementasikan dalam fungsi flatten().
\end{eulercomment}
\begin{eulerprompt}
>redim(A,1,prod(size(A))), flatten(A)
\end{eulerprompt}
\begin{euleroutput}
  [1,  2,  3,  4,  5,  6,  7,  8,  9]
  [1,  2,  3,  4,  5,  6,  7,  8,  9]
\end{euleroutput}
\begin{eulercomment}
Untuk menggunakan matriks pada tabel, mari kita atur ulang ke format
default, dan hitung tabel nilai sinus dan kosinus. Perhatikan bahwa
sudut dalam radian secara default.
\end{eulercomment}
\begin{eulerprompt}
>defformat; w=0°:45°:360°; w=w'; deg(w)
\end{eulerprompt}
\begin{euleroutput}
              0 
             45 
             90 
            135 
            180 
            225 
            270 
            315 
            360 
\end{euleroutput}
\begin{eulercomment}
Sekarang kita tambahkan kolom ke matriks.
\end{eulercomment}
\begin{eulerprompt}
>M = deg(w)|w|cos(w)|sin(w)
\end{eulerprompt}
\begin{euleroutput}
              0             0             1             0 
             45      0.785398      0.707107      0.707107 
             90        1.5708             0             1 
            135       2.35619     -0.707107      0.707107 
            180       3.14159            -1             0 
            225       3.92699     -0.707107     -0.707107 
            270       4.71239             0            -1 
            315       5.49779      0.707107     -0.707107 
            360       6.28319             1             0 
\end{euleroutput}
\begin{eulercomment}
Dengan menggunakan bahasa matriks, kita dapat membuat beberapa tabel
dari beberapa fungsi sekaligus.

Dalam contoh berikut, kita menghitung t[j]\textasciicircum{}i untuk i dari 1 hingga n.
Kita memperoleh matriks, yang setiap barisnya merupakan tabel t\textasciicircum{}i
untuk satu i. Yaitu, matriks tersebut memiliki elemen lateks: a\_\{i,j\}
= t\_j\textasciicircum{}i, \textbackslash{}quad 1 \textbackslash{}le j \textbackslash{}le 101, \textbackslash{}quad 1 \textbackslash{}le i \textbackslash{}le n

Fungsi yang tidak berfungsi untuk input vektor harus "divektorkan".
Ini dapat dicapai dengan kata kunci "map" dalam definisi fungsi.
Kemudian fungsi tersebut akan dievaluasi untuk setiap elemen parameter
vektor.

Integrasi numerik integr() hanya berfungsi untuk batas interval
skalar. Jadi, kita perlu memvektorkannya.
\end{eulercomment}
\begin{eulerprompt}
>function map f(x) := integrate("x^x",1,x)
\end{eulerprompt}
\begin{eulercomment}
Kata kunci "map" akan memvektorkan fungsi tersebut. Fungsi tersebut
sekarang akan berfungsi\\
untuk vektor angka.
\end{eulercomment}
\begin{eulerprompt}
>f([1:5])
\end{eulerprompt}
\begin{euleroutput}
  [0,  2.05045,  13.7251,  113.336,  1241.03]
\end{euleroutput}
\eulerheading{Sub-Matriks dan Elemen Matriks}
\begin{eulercomment}
Untuk mengakses elemen matriks, gunakan notasi tanda kurung.
\end{eulercomment}
\begin{eulerprompt}
>A=[1,2,3;4,5,6;7,8,9], A[2,2]
\end{eulerprompt}
\begin{euleroutput}
              1             2             3 
              4             5             6 
              7             8             9 
  5
\end{euleroutput}
\begin{eulercomment}
Kita dapat mengakses baris matriks yang lengkap.
\end{eulercomment}
\begin{eulerprompt}
>A[2]
\end{eulerprompt}
\begin{euleroutput}
  [4,  5,  6]
\end{euleroutput}
\begin{eulercomment}
Dalam kasus vektor baris atau kolom, ini mengembalikan elemen vektor.
\end{eulercomment}
\begin{eulerprompt}
>v=1:3; v[2]
\end{eulerprompt}
\begin{euleroutput}
  2
\end{euleroutput}
\begin{eulercomment}
Untuk memastikan, Anda mendapatkan baris pertama untuk matriks 1xn dan
mxn, tentukan semua kolom menggunakan indeks kedua yang kosong.
\end{eulercomment}
\begin{eulerprompt}
>A[2,]
\end{eulerprompt}
\begin{euleroutput}
  [4,  5,  6]
\end{euleroutput}
\begin{eulercomment}
Jika indeks adalah vektor indeks, Euler akan mengembalikan baris
matriks yang sesuai.

Di sini kita menginginkan baris pertama dan kedua dari A.
\end{eulercomment}
\begin{eulerprompt}
>A[[1,2]]
\end{eulerprompt}
\begin{euleroutput}
              1             2             3 
              4             5             6 
\end{euleroutput}
\begin{eulercomment}
Kita bahkan dapat menyusun ulang A menggunakan vektor indeks. Untuk
lebih tepatnya, kita tidak mengubah A di sini, tetapi menghitung versi
A yang telah disusun ulang.
\end{eulercomment}
\begin{eulerprompt}
>A[[3,2,1]]
\end{eulerprompt}
\begin{euleroutput}
              7             8             9 
              4             5             6 
              1             2             3 
\end{euleroutput}
\begin{eulercomment}
Trik indeks juga berfungsi dengan kolom.

Contoh ini memilih semua baris A dan kolom kedua dan ketiga.
\end{eulercomment}
\begin{eulerprompt}
>A[1:3,2:3]
\end{eulerprompt}
\begin{euleroutput}
              2             3 
              5             6 
              8             9 
\end{euleroutput}
\begin{eulercomment}
Untuk singkatan ":" menunjukkan semua indeks baris atau kolom.
\end{eulercomment}
\begin{eulerprompt}
>A[:,3]
\end{eulerprompt}
\begin{euleroutput}
              3 
              6 
              9 
\end{euleroutput}
\begin{eulercomment}
Atau, biarkan indeks pertama kosong.
\end{eulercomment}
\begin{eulerprompt}
>A[,2:3]
\end{eulerprompt}
\begin{euleroutput}
              2             3 
              5             6 
              8             9 
\end{euleroutput}
\begin{eulercomment}
Kita juga bisa mendapatkan baris terakhir A.
\end{eulercomment}
\begin{eulerprompt}
>A[-1]
\end{eulerprompt}
\begin{euleroutput}
  [7,  8,  9]
\end{euleroutput}
\begin{eulercomment}
Sekarang mari kita ubah elemen A dengan menetapkan submatriks A ke
suatu nilai. Hal ini pada kenyataannya mengubah matriks A yang
tersimpan.
\end{eulercomment}
\begin{eulerprompt}
>A[1,1]=4
\end{eulerprompt}
\begin{euleroutput}
              4             2             3 
              4             5             6 
              7             8             9 
\end{euleroutput}
\begin{eulercomment}
Kita juga dapat menetapkan nilai ke baris A.
\end{eulercomment}
\begin{eulerprompt}
>A[1]=[-1,-1,-1]
\end{eulerprompt}
\begin{euleroutput}
             -1            -1            -1 
              4             5             6 
              7             8             9 
\end{euleroutput}
\begin{eulercomment}
Kita bahkan dapat menetapkannya ke submatriks jika ukurannya tepat.
\end{eulercomment}
\begin{eulerprompt}
>A[1:2,1:2]=[5,6;7,8]
\end{eulerprompt}
\begin{euleroutput}
              5             6            -1 
              7             8             6 
              7             8             9 
\end{euleroutput}
\begin{eulercomment}
Selain itu, beberapa jalan pintas diperbolehkan.
\end{eulercomment}
\begin{eulerprompt}
>A[1:2,1:2]=0
\end{eulerprompt}
\begin{euleroutput}
              0             0            -1 
              0             0             6 
              7             8             9 
\end{euleroutput}
\begin{eulercomment}
Peringatan: Indeks yang tidak sesuai batas akan mengembalikan matriks
kosong, atau pesan kesalahan, tergantung pada pengaturan sistem. Pesan
kesalahan adalah standar. Namun, perlu diingat bahwa indeks negatif
dapat digunakan untuk mengakses elemen matriks yang dihitung dari
akhir.
\end{eulercomment}
\begin{eulerprompt}
>A[3]
\end{eulerprompt}
\begin{euleroutput}
  [7,  8,  9]
\end{euleroutput}
\eulerheading{Sortir dan Acak}
\begin{eulercomment}
Fungsi sort() mengurutkan vektor baris.
\end{eulercomment}
\begin{eulerprompt}
>sort([5,6,4,8,1,9])
\end{eulerprompt}
\begin{euleroutput}
  [1,  4,  5,  6,  8,  9]
\end{euleroutput}
\begin{eulercomment}
Seringkali perlu untuk mengetahui indeks vektor yang diurutkan dalam
vektor asli. Ini dapat digunakan untuk menyusun ulang vektor lain
dengan cara yang sama.

Mari kita acak sebuah vektor.
\end{eulercomment}
\begin{eulerprompt}
>v=shuffle(1:10)
\end{eulerprompt}
\begin{euleroutput}
  [4,  5,  10,  6,  8,  9,  1,  7,  2,  3]
\end{euleroutput}
\begin{eulercomment}
Indeks berisi urutan v yang tepat.
\end{eulercomment}
\begin{eulerprompt}
>\{vs,ind\}=sort(v); v[ind]
\end{eulerprompt}
\begin{euleroutput}
  [1,  2,  3,  4,  5,  6,  7,  8,  9,  10]
\end{euleroutput}
\begin{eulercomment}
Ini juga berlaku untuk vektor string.
\end{eulercomment}
\begin{eulerprompt}
>s=["a","d","e","a","aa","e"]
\end{eulerprompt}
\begin{euleroutput}
  a
  d
  e
  a
  aa
  e
\end{euleroutput}
\begin{eulerprompt}
>\{ss,ind\}=sort(s); ss
\end{eulerprompt}
\begin{euleroutput}
  a
  a
  aa
  d
  e
  e
\end{euleroutput}
\begin{eulercomment}
Seperti yang Anda lihat, posisi entri ganda agak acak.
\end{eulercomment}
\begin{eulerprompt}
>ind
\end{eulerprompt}
\begin{euleroutput}
  [4,  1,  5,  2,  6,  3]
\end{euleroutput}
\begin{eulercomment}
Fungsi unik mengembalikan daftar yang diurutkan dari elemen unik suatu
vektor.
\end{eulercomment}
\begin{eulerprompt}
>intrandom(1,10,10), unique(%)
\end{eulerprompt}
\begin{euleroutput}
  [4,  4,  9,  2,  6,  5,  10,  6,  5,  1]
  [1,  2,  4,  5,  6,  9,  10]
\end{euleroutput}
\begin{eulercomment}
Ini juga berlaku untuk vektor string.
\end{eulercomment}
\begin{eulerprompt}
>unique(s)
\end{eulerprompt}
\begin{euleroutput}
  a
  aa
  d
  e
\end{euleroutput}
\eulerheading{Aljabar Linier}
\begin{eulercomment}
EMT memiliki banyak fungsi untuk memecahkan sistem linier, sistem
renggang, atau masalah regresi.

Untuk sistem linier Ax=b, Anda dapat menggunakan algoritma Gauss,
matriks invers, atau kecocokan linier. Operator A\textbackslash{}b menggunakan versi
algoritma Gauss.
\end{eulercomment}
\begin{eulerprompt}
>A=[1,2;3,4]; b=[5;6]; A\(\backslash\)b
\end{eulerprompt}
\begin{euleroutput}
             -4 
            4.5 
\end{euleroutput}
\begin{eulercomment}
Untuk contoh lain, kita buat matriks 200x200 dan jumlah barisnya.
Kemudian kita selesaikan Ax=b menggunakan matriks invers. Kita ukur
kesalahan sebagai deviasi maksimal semua elemen dari 1, yang tentu
saja merupakan solusi yang benar.
\end{eulercomment}
\begin{eulerprompt}
>A=normal(200,200); b=sum(A); longest totalmax(abs(inv(A).b-1))
\end{eulerprompt}
\begin{euleroutput}
    8.790745908981989e-13 
\end{euleroutput}
\begin{eulercomment}
Jika sistem tidak mempunyai solusi, penyesuaian linier meminimalkan
norma kesalahan Ax-b.
\end{eulercomment}
\begin{eulerprompt}
>A=[1,2,3;4,5,6;7,8,9]
\end{eulerprompt}
\begin{euleroutput}
              1             2             3 
              4             5             6 
              7             8             9 
\end{euleroutput}
\begin{eulercomment}
Determinan matriks ini adalah 0.
\end{eulercomment}
\begin{eulerprompt}
>det(A)
\end{eulerprompt}
\begin{euleroutput}
  0
\end{euleroutput}
\eulerheading{Matriks Simbolik}
\begin{eulercomment}
Maxima memiliki matriks simbolik. Tentu saja, Maxima dapat digunakan
untuk masalah aljabar linear sederhana tersebut. Kita dapat
mendefinisikan matriks untuk Euler dan Maxima dengan \&:=, lalu
menggunakannya dalam ekspresi simbolik. Bentuk [...] yang biasa
digunakan untuk mendefinisikan matriks dapat digunakan dalam Euler
untuk mendefinisikan matriks simbolik.
\end{eulercomment}
\begin{eulerprompt}
>A &= [a,1,1;1,a,1;1,1,a]; $A
\end{eulerprompt}
\begin{eulerformula}
\[
\begin{pmatrix}a & 1 & 1 \\ 1 & a & 1 \\ 1 & 1 & a \\ \end{pmatrix}
\]
\end{eulerformula}
\begin{eulerprompt}
>$&det(A), $&factor(%)
\end{eulerprompt}
\begin{eulerformula}
\[
a\,\left(a^2-1\right)-2\,a+2
\]
\end{eulerformula}
\begin{eulerformula}
\[
\left(a-1\right)^2\,\left(a+2\right)
\]
\end{eulerformula}
\begin{eulerprompt}
>$&invert(A) with a=0
\end{eulerprompt}
\begin{eulerformula}
\[
\begin{pmatrix}-\frac{1}{2} & \frac{1}{2} & \frac{1}{2} \\ \frac{1
 }{2} & -\frac{1}{2} & \frac{1}{2} \\ \frac{1}{2} & \frac{1}{2} & -
 \frac{1}{2} \\ \end{pmatrix}
\]
\end{eulerformula}
\begin{eulerprompt}
>A &= [1,a;b,2]; $A
\end{eulerprompt}
\begin{eulerformula}
\[
\begin{pmatrix}1 & a \\ b & 2 \\ \end{pmatrix}
\]
\end{eulerformula}
\begin{eulercomment}
Seperti semua variabel simbolik, matriks ini dapat digunakan dalam
ekspresi simbolik lainnya.
\end{eulercomment}
\begin{eulerprompt}
>$&det(A-x*ident(2)), $&solve(%,x)
\end{eulerprompt}
\begin{eulerformula}
\[
\left(1-x\right)\,\left(2-x\right)-a\,b
\]
\end{eulerformula}
\begin{eulerformula}
\[
\left[ x=\frac{3-\sqrt{4\,a\,b+1}}{2} , x=\frac{\sqrt{4\,a\,b+1}+3
 }{2} \right] 
\]
\end{eulerformula}
\begin{eulercomment}
Nilai eigen juga dapat dihitung secara otomatis. Hasilnya adalah
vektor dengan dua vektor nilai eigen dan multiplisitas.
\end{eulercomment}
\begin{eulerprompt}
>$&eigenvalues([a,1;1,a])
\end{eulerprompt}
\begin{eulerformula}
\[
\left[ \left[ a-1 , a+1 \right]  , \left[ 1 , 1 \right]  \right] 
\]
\end{eulerformula}
\begin{eulercomment}
Untuk mengekstrak vektor eigen tertentu dibutuhkan pengindeksan yang
cermat.
\end{eulercomment}
\begin{eulerprompt}
>$&eigenvectors([a,1;1,a]), &%[2][1][1]
\end{eulerprompt}
\begin{eulerformula}
\[
\left[ \left[ \left[ a-1 , a+1 \right]  , \left[ 1 , 1 \right] 
  \right]  , \left[ \left[ \left[ 1 , -1 \right]  \right]  , \left[ 
 \left[ 1 , 1 \right]  \right]  \right]  \right] 
\]
\end{eulerformula}
\begin{euleroutput}
  
                                 [1, - 1]
  
\end{euleroutput}
\begin{eulercomment}
Matriks simbolik dapat dievaluasi dalam Euler secara numerik seperti
ekspresi simbolik lainnya.
\end{eulercomment}
\begin{eulerprompt}
>A(a=4,b=5)
\end{eulerprompt}
\begin{euleroutput}
              1             4 
              5             2 
\end{euleroutput}
\begin{eulercomment}
Dalam ekspresi simbolik, gunakan dengan.
\end{eulercomment}
\begin{eulerprompt}
>$&A with [a=4,b=5]
\end{eulerprompt}
\begin{eulerformula}
\[
\begin{pmatrix}1 & 4 \\ 5 & 2 \\ \end{pmatrix}
\]
\end{eulerformula}
\begin{eulercomment}
Akses terhadap baris matriks simbolik bekerja seperti halnya matriks
numerik.
\end{eulercomment}
\begin{eulerprompt}
>$&A[1]
\end{eulerprompt}
\begin{eulerformula}
\[
\left[ 1 , a \right] 
\]
\end{eulerformula}
\begin{eulercomment}
Ekspresi simbolik dapat berisi sebuah penugasan. Dan itu mengubah
matriks A.
\end{eulercomment}
\begin{eulerprompt}
>&A[1,1]:=t+1; $&A
\end{eulerprompt}
\begin{eulerformula}
\[
\begin{pmatrix}t+1 & a \\ b & 2 \\ \end{pmatrix}
\]
\end{eulerformula}
\begin{eulercomment}
Terdapat fungsi simbolik di Maxima untuk membuat vektor dan matriks.
Untuk ini, rujuk dokumentasi Maxima atau tutorial tentang Maxima di
EMT.
\end{eulercomment}
\begin{eulerprompt}
>v &= makelist(1/(i+j),i,1,3); $v
\end{eulerprompt}
\begin{eulerformula}
\[
\left[ \frac{1}{j+1} , \frac{1}{j+2} , \frac{1}{j+3} \right] 
\]
\end{eulerformula}
\begin{eulerprompt}
>B &:= [1,2;3,4]; $B, $&invert(B)
\end{eulerprompt}
\begin{eulerformula}
\[
\begin{pmatrix}1 & 2 \\ 3 & 4 \\ \end{pmatrix}
\]
\end{eulerformula}
\begin{eulerformula}
\[
\begin{pmatrix}-2 & 1 \\ \frac{3}{2} & -\frac{1}{2} \\ 
 \end{pmatrix}
\]
\end{eulerformula}
\begin{eulercomment}
Hasilnya dapat dievaluasi secara numerik di Euler. Untuk informasi
lebih lanjut tentang Maxima, lihat pengantar Maxima.
\end{eulercomment}
\begin{eulerprompt}
>$&invert(B)()
\end{eulerprompt}
\begin{euleroutput}
             -2             1 
            1.5          -0.5 
\end{euleroutput}
\begin{eulercomment}
Euler juga memiliki fungsi xinv() yang hebat, yang melakukan upaya
lebih besar dan mendapatkan hasil yang lebih tepat.

Perlu dicatat, bahwa dengan \&:= matriks B telah didefinisikan sebagai
simbolik dalam ekspresi simbolik dan sebagai numerik dalam ekspresi
numerik. Jadi kita dapat menggunakannya di sini.
\end{eulercomment}
\begin{eulerprompt}
>longest B.xinv(B)
\end{eulerprompt}
\begin{euleroutput}
                        1                       0 
                        0                       1 
\end{euleroutput}
\begin{eulercomment}
Misalnya nilai eigen A dapat dihitung secara numerik.
\end{eulercomment}
\begin{eulerprompt}
>A=[1,2,3;4,5,6;7,8,9]; real(eigenvalues(A))
\end{eulerprompt}
\begin{euleroutput}
  [16.1168,  -1.11684,  0]
\end{euleroutput}
\begin{eulercomment}
Atau secara simbolis. Lihat tutorial tentang Maxima untuk detailnya.
\end{eulercomment}
\begin{eulerprompt}
>$&eigenvalues(@A)
\end{eulerprompt}
\begin{eulerformula}
\[
\left[ \left[ \frac{15-3\,\sqrt{33}}{2} , \frac{3\,\sqrt{33}+15}{2}
  , 0 \right]  , \left[ 1 , 1 , 1 \right]  \right] 
\]
\end{eulerformula}
\eulerheading{Nilai Numerik dalam Ekspresi Simbolik}
\begin{eulercomment}
Ekspresi simbolik hanyalah string yang berisi ekspresi. Jika kita
ingin menentukan nilai untuk ekspresi simbolik dan ekspresi numerik,
kita harus menggunakan "\&:=".
\end{eulercomment}
\begin{eulerprompt}
>A &:= [1,pi;4,5]
\end{eulerprompt}
\begin{euleroutput}
              1       3.14159 
              4             5 
\end{euleroutput}
\begin{eulercomment}
Masih terdapat perbedaan antara bentuk numerik dan bentuk simbolik.
Saat mengubah matriks ke bentuk simbolik, pendekatan pecahan untuk
bilangan riil akan digunakan.
\end{eulercomment}
\begin{eulerprompt}
>$&A
\end{eulerprompt}
\begin{eulerformula}
\[
\begin{pmatrix}1 & \frac{1146408}{364913} \\ 4 & 5 \\ \end{pmatrix}
\]
\end{eulerformula}
\begin{eulercomment}
Untuk menghindari hal ini, ada fungsi "mxmset(variabel)".
\end{eulercomment}
\begin{eulerprompt}
>mxmset(A); $&A
\end{eulerprompt}
\begin{eulerformula}
\[
\begin{pmatrix}1 & 3.141592653589793 \\ 4 & 5 \\ \end{pmatrix}
\]
\end{eulerformula}
\begin{eulercomment}
Maxima juga dapat melakukan komputasi dengan angka floating point, dan
bahkan dengan angka floating point besar dengan 32 digit. Namun,
evaluasinya jauh lebih lambat.
\end{eulercomment}
\begin{eulerprompt}
>$&bfloat(sqrt(2)), $&float(sqrt(2))
\end{eulerprompt}
\begin{eulerformula}
\[
1.4142135623730950488016887242097_B \times 10^{0}
\]
\end{eulerformula}
\begin{eulerformula}
\[
1.414213562373095
\]
\end{eulerformula}
\begin{eulercomment}
Ketepatan angka floating point besar dapat diubah.
\end{eulercomment}
\begin{eulerprompt}
>&fpprec:=100; &bfloat(pi)
\end{eulerprompt}
\begin{euleroutput}
  
          3.14159265358979323846264338327950288419716939937510582097494\(\backslash\)
  4592307816406286208998628034825342117068b0
  
\end{euleroutput}
\begin{eulercomment}
Variabel numerik dapat digunakan dalam ekspresi simbolik apa pun
menggunakan "@var".

Perlu dicatat bahwa ini hanya diperlukan jika variabel telah
didefinisikan dengan ":=" atau "=" sebagai variabel numerik.
\end{eulercomment}
\begin{eulerprompt}
>B:=[1,pi;3,4]; $&det(@B)
\end{eulerprompt}
\begin{eulerformula}
\[
-5.424777960769379
\]
\end{eulerformula}
\eulerheading{Demo - Suku Bunga}
\begin{eulercomment}
Di bawah ini, kami menggunakan Euler Math Toolbox (EMT) untuk
menghitung suku bunga. Kami melakukannya secara numerik dan simbolis
untuk menunjukkan kepada Anda bagaimana Euler dapat digunakan untuk
memecahkan masalah kehidupan nyata.

Asumsikan Anda memiliki modal awal sebesar 5000 (misalnya dalam
dolar).
\end{eulercomment}
\begin{eulerprompt}
>K=5000
\end{eulerprompt}
\begin{euleroutput}
  5000
\end{euleroutput}
\begin{eulercomment}
Sekarang kita asumsikan suku bunga 3\% per tahun. Mari kita tambahkan
satu suku bunga sederhana dan hitung hasilnya.
\end{eulercomment}
\begin{eulerprompt}
>K*1.03
\end{eulerprompt}
\begin{euleroutput}
  5150
\end{euleroutput}
\begin{eulercomment}
Euler juga akan memahami sintaks berikut.
\end{eulercomment}
\begin{eulerprompt}
>K+K*3%
\end{eulerprompt}
\begin{euleroutput}
  5150
\end{euleroutput}
\begin{eulercomment}
Namun lebih mudah menggunakan faktor
\end{eulercomment}
\begin{eulerprompt}
>q=1+3%, K*q
\end{eulerprompt}
\begin{euleroutput}
  1.03
  5150
\end{euleroutput}
\begin{eulercomment}
Selama 10 tahun, kita cukup mengalikan faktor-faktornya dan
mendapatkan nilai akhir dengan suku bunga majemuk.
\end{eulercomment}
\begin{eulerprompt}
>K*q^10
\end{eulerprompt}
\begin{euleroutput}
  6719.58189672
\end{euleroutput}
\begin{eulercomment}
Untuk keperluan kita, kita dapat mengatur format menjadi 2 digit
setelah titik desimal.
\end{eulercomment}
\begin{eulerprompt}
>format(12,2); K*q^10
\end{eulerprompt}
\begin{euleroutput}
      6719.58 
\end{euleroutput}
\begin{eulercomment}
Mari kita cetak angka tersebut dibulatkan menjadi 2 digit dalam
kalimat lengkap.
\end{eulercomment}
\begin{eulerprompt}
>"Starting from " + K + "$ you get " + round(K*q^10,2) + "$."
\end{eulerprompt}
\begin{euleroutput}
  Starting from 5000$ you get 6719.58$.
\end{euleroutput}
\begin{eulercomment}
Bagaimana jika kita ingin mengetahui hasil antara dari tahun 1 hingga
tahun 9? Untuk ini, bahasa matriks Euler sangat membantu. Anda tidak
perlu menulis loop, tetapi cukup masukkan
\end{eulercomment}
\begin{eulerprompt}
>K*q^(0:10)
\end{eulerprompt}
\begin{euleroutput}
  Real 1 x 11 matrix
  
      5000.00     5150.00     5304.50     5463.64     ...
\end{euleroutput}
\begin{eulercomment}
Bagaimana keajaiban ini bekerja? Pertama, ekspresi 0:10 menghasilkan
vektor bilangan bulat.
\end{eulercomment}
\begin{eulerprompt}
>short 0:10
\end{eulerprompt}
\begin{euleroutput}
  [0,  1,  2,  3,  4,  5,  6,  7,  8,  9,  10]
\end{euleroutput}
\begin{eulercomment}
Maka semua operator dan fungsi di Euler dapat diaplikasikan ke vektor
elemen demi elemen. Jadi
\end{eulercomment}
\begin{eulerprompt}
>short q^(0:10)
\end{eulerprompt}
\begin{euleroutput}
  [1,  1.03,  1.0609,  1.0927,  1.1255,  1.1593,  1.1941,  1.2299,
  1.2668,  1.3048,  1.3439]
\end{euleroutput}
\begin{eulercomment}
adalah vektor faktor q\textasciicircum{}0 hingga q\textasciicircum{}10. Ini dikalikan dengan K, dan kita
memperoleh vektor nilai.
\end{eulercomment}
\begin{eulerprompt}
>VK=K*q^(0:10);
\end{eulerprompt}
\begin{eulercomment}
Tentu saja, cara realistis untuk menghitung suku bunga ini adalah
dengan membulatkannya ke sen terdekat setelah setiap tahun. Mari kita
tambahkan fungsi untuk ini.
\end{eulercomment}
\begin{eulerprompt}
>function oneyear (K) := round(K*q,2)
\end{eulerprompt}
\begin{eulercomment}
Mari kita bandingkan kedua hasil, dengan dan tanpa pembulatan.
\end{eulercomment}
\begin{eulerprompt}
>longest oneyear(1234.57), longest 1234.57*q
\end{eulerprompt}
\begin{euleroutput}
                  1271.61 
                1271.6071 
\end{euleroutput}
\begin{eulercomment}
Sekarang tidak ada rumus sederhana untuk tahun ke-n, dan kita harus
mengulangnya selama bertahun-tahun. Euler menyediakan banyak solusi
untuk ini.

Cara termudah adalah fungsi iterate, yang mengulang fungsi yang
diberikan beberapa kali.
\end{eulercomment}
\begin{eulerprompt}
>VKr=iterate("oneyear",5000,10)
\end{eulerprompt}
\begin{euleroutput}
  Real 1 x 11 matrix
  
      5000.00     5150.00     5304.50     5463.64     ...
\end{euleroutput}
\begin{eulercomment}
Kita dapat mencetaknya dengan cara yang ramah, menggunakan format kami
dengan tempat desimal tetap.
\end{eulercomment}
\begin{eulerprompt}
>VKr'
\end{eulerprompt}
\begin{euleroutput}
      5000.00 
      5150.00 
      5304.50 
      5463.64 
      5627.55 
      5796.38 
      5970.27 
      6149.38 
      6333.86 
      6523.88 
      6719.60 
\end{euleroutput}
\begin{eulercomment}
Untuk mendapatkan elemen vektor tertentu, kita menggunakan indeks
dalam tanda kurung siku.
\end{eulercomment}
\begin{eulerprompt}
>VKr[2], VKr[1:3]
\end{eulerprompt}
\begin{euleroutput}
      5150.00 
      5000.00     5150.00     5304.50 
\end{euleroutput}
\begin{eulercomment}
Anehnya, kita juga dapat menggunakan vektor indeks. Ingat bahwa 1:3
menghasilkan vektor [1,2,3].

Mari kita bandingkan elemen terakhir dari nilai yang dibulatkan dengan
nilai penuh.
\end{eulercomment}
\begin{eulerprompt}
>VKr[-1], VK[-1]
\end{eulerprompt}
\begin{euleroutput}
      6719.60 
      6719.58 
\end{euleroutput}
\begin{eulercomment}
Perbedaannya sangat kecil.

\begin{eulercomment}
\eulerheading{Menyelesaikan Persamaan}
\begin{eulercomment}
Sekarang kita ambil fungsi yang lebih maju, yang menambahkan nilai
uang tertentu setiap tahun.
\end{eulercomment}
\begin{eulerprompt}
>function onepay (K) := K*q+R
\end{eulerprompt}
\begin{eulercomment}
Kita tidak perlu menentukan q atau R untuk definisi fungsi. Hanya jika
kita menjalankan perintah, kita harus menentukan nilai-nilai ini. Kita
pilih R=200.
\end{eulercomment}
\begin{eulerprompt}
>R=200; iterate("onepay",5000,10)
\end{eulerprompt}
\begin{euleroutput}
  Real 1 x 11 matrix
  
      5000.00     5350.00     5710.50     6081.82     ...
\end{euleroutput}
\begin{eulercomment}
Bagaimana jika kita menghilangkan jumlah yang sama setiap tahun?
\end{eulercomment}
\begin{eulerprompt}
>R=-200; iterate("onepay",5000,10)
\end{eulerprompt}
\begin{euleroutput}
  Real 1 x 11 matrix
  
      5000.00     4950.00     4898.50     4845.45     ...
\end{euleroutput}
\begin{eulercomment}
Kita melihat bahwa uang berkurang. Jelas, jika kita hanya memperoleh
bunga sebesar 150 pada tahun pertama, tetapi mengurangi 200, kita akan
kehilangan uang setiap tahun.

Bagaimana kita dapat menentukan berapa tahun uang tersebut akan
bertahan? Kita harus menulis sebuah loop untuk ini. Cara termudah
adalah dengan melakukan iterasi yang cukup lama.
\end{eulercomment}
\begin{eulerprompt}
>VKR=iterate("onepay",5000,50)
\end{eulerprompt}
\begin{euleroutput}
  Real 1 x 51 matrix
  
      5000.00     4950.00     4898.50     4845.45     ...
\end{euleroutput}
\begin{eulercomment}
Dengan menggunakan bahasa matriks, kita dapat menentukan nilai negatif
pertama dengan cara berikut.
\end{eulercomment}
\begin{eulerprompt}
>min(nonzeros(VKR<0))
\end{eulerprompt}
\begin{euleroutput}
        48.00 
\end{euleroutput}
\begin{eulercomment}
Alasannya adalah nonzeros(VKR\textless{}0) mengembalikan vektor indeks i, di
mana VKR[i]\textless{}0, dan min menghitung indeks minimal.

Karena vektor selalu dimulai dengan indeks 1, jawabannya adalah 47
tahun.

Fungsi iterate() memiliki satu trik lagi. Fungsi ini dapat mengambil
kondisi akhir sebagai argumen. Kemudian, fungsi ini akan mengembalikan
nilai dan jumlah iterasi.
\end{eulercomment}
\begin{eulerprompt}
>\{x,n\}=iterate("onepay",5000,till="x<0"); x, n,
\end{eulerprompt}
\begin{euleroutput}
       -19.83 
        47.00 
\end{euleroutput}
\begin{eulercomment}
Mari kita coba menjawab pertanyaan yang lebih ambigu. Asumsikan kita
tahu bahwa nilainya adalah 0 setelah 50 tahun. Berapa tingkat
bunganya?

Ini adalah pertanyaan yang hanya dapat dijawab secara numerik. Di
bawah ini, kita akan memperoleh rumus yang diperlukan. Kemudian Anda
akan melihat bahwa tidak ada rumus yang mudah untuk tingkat bunga.
Namun untuk saat ini, kita bertujuan untuk mencari solusi numerik.

Langkah pertama adalah mendefinisikan fungsi yang melakukan iterasi
sebanyak n kali. Kita menambahkan semua parameter ke fungsi ini.
\end{eulercomment}
\begin{eulerprompt}
>function f(K,R,P,n) := iterate("x*(1+P/100)+R",K,n;P,R)[-1]
\end{eulerprompt}
\begin{eulercomment}
Iterasinya sama seperti di atas

\end{eulercomment}
\begin{eulerformula}
\[
x_{n+1} = x_n \cdot \left(1+ \frac{P}{100}\right) + R
\]
\end{eulerformula}
\begin{eulercomment}
Namun, kita tidak lagi menggunakan nilai global R dalam ekspresi kita.
Fungsi seperti iterate() memiliki trik khusus di Euler. Anda dapat
meneruskan nilai variabel dalam ekspresi sebagai parameter titik koma.
Dalam kasus ini P dan R.

Selain itu, kita hanya tertarik pada nilai terakhir. Jadi, kita ambil
indeks [-1].

Mari kita coba uji coba.
\end{eulercomment}
\begin{eulerprompt}
>f(5000,-200,3,47)
\end{eulerprompt}
\begin{euleroutput}
       -19.83 
\end{euleroutput}
\begin{eulercomment}
Sekarang kita bisa memecahkan masalah kita.
\end{eulercomment}
\begin{eulerprompt}
>solve("f(5000,-200,x,50)",3)
\end{eulerprompt}
\begin{euleroutput}
         3.15 
\end{euleroutput}
\begin{eulercomment}
Rutin solve menyelesaikan ekspresi=0 untuk variabel x. Jawabannya
adalah 3,15\% per tahun. Kita ambil nilai awal 3\% untuk algoritma
tersebut. Fungsi solve() selalu membutuhkan nilai awal.

Kita dapat menggunakan fungsi yang sama untuk menyelesaikan pertanyaan
berikut: Berapa banyak yang dapat kita hapus per tahun sehingga modal
awal habis setelah 20 tahun dengan asumsi suku bunga 3\% per tahun.
\end{eulercomment}
\begin{eulerprompt}
>solve("f(5000,x,3,20)",-200)
\end{eulerprompt}
\begin{euleroutput}
      -336.08 
\end{euleroutput}
\begin{eulercomment}
Perhatikan bahwa Anda tidak dapat memecahkan masalah jumlah tahun,
karena fungsi kita mengasumsikan n sebagai nilai integer.

\end{eulercomment}
\eulersubheading{Solusi Simbolis untuk Masalah Suku Bunga}
\begin{eulercomment}
Kita dapat menggunakan bagian simbolis Euler untuk mempelajari masalah
tersebut. Pertama, kita mendefinisikan fungsi onepay() secara
simbolis.
\end{eulercomment}
\begin{eulerprompt}
>function op(K) &= K*q+R; $&op(K)
\end{eulerprompt}
\begin{eulerformula}
\[
R+q\,K
\]
\end{eulerformula}
\begin{eulercomment}
Sekarang kita dapat mengulanginya.
\end{eulercomment}
\begin{eulerprompt}
>$&op(op(op(op(K)))), $&expand(%)
\end{eulerprompt}
\begin{eulerformula}
\[
q\,\left(q\,\left(q\,\left(R+q\,K\right)+R\right)+R\right)+R
\]
\end{eulerformula}
\begin{eulerformula}
\[
q^3\,R+q^2\,R+q\,R+R+q^4\,K
\]
\end{eulerformula}
\begin{eulercomment}
Kita melihat suatu pola. Setelah n periode kita memiliki

\end{eulercomment}
\begin{eulerformula}
\[
K_n = q^n K + R (1+q+\ldots+q^{n-1}) = q^n K + \frac{q^n-1}{q-1} R
\]
\end{eulerformula}
\begin{eulercomment}
Rumus tersebut adalah rumus untuk jumlah geometrik, yang diketahui
oleh Maxima.
\end{eulercomment}
\begin{eulerprompt}
>&sum(q^k,k,0,n-1); $& % = ev(%,simpsum)
\end{eulerprompt}
\begin{eulerformula}
\[
\sum_{k=0}^{n-1}{q^{k}}=\frac{q^{n}-1}{q-1}
\]
\end{eulerformula}
\begin{eulercomment}
Ini agak rumit. Jumlahnya dievaluasi dengan tanda "simpsum" untuk
mereduksinya menjadi hasil bagi.

Mari kita buat fungsi untuk ini.
\end{eulercomment}
\begin{eulerprompt}
>function fs(K,R,P,n) &= (1+P/100)^n*K + ((1+P/100)^n-1)/(P/100)*R; $&fs(K,R,P,n)
\end{eulerprompt}
\begin{eulerformula}
\[
\frac{100\,\left(\left(\frac{P}{100}+1\right)^{n}-1\right)\,R}{P}+K
 \,\left(\frac{P}{100}+1\right)^{n}
\]
\end{eulerformula}
\begin{eulercomment}
Fungsi ini melakukan hal yang sama seperti fungsi f sebelumnya. Namun,
fungsinya lebih efektif.
\end{eulercomment}
\begin{eulerprompt}
>longest f(5000,-200,3,47), longest fs(5000,-200,3,47)
\end{eulerprompt}
\begin{euleroutput}
       -19.82504734650985 
       -19.82504734652684 
\end{euleroutput}
\begin{eulercomment}
Kita sekarang dapat menggunakannya untuk menanyakan waktu n. Kapan
modal kita habis? Perkiraan awal kita adalah 30 tahun.
\end{eulercomment}
\begin{eulerprompt}
>solve("fs(5000,-330,3,x)",30)
\end{eulerprompt}
\begin{euleroutput}
        20.51 
\end{euleroutput}
\begin{eulercomment}
Jawaban ini menyatakan bahwa akan negatif setelah 21 tahun.

Kita juga dapat menggunakan sisi simbolik Euler untuk menghitung rumus
pembayaran.

Asumsikan kita mendapatkan pinjaman sebesar K, dan membayar n kali
cicilan sebesar R (dimulai setelah tahun pertama) sehingga menyisakan
utang residual sebesar Kn (pada saat pembayaran terakhir). Rumus untuk
ini jelas
\end{eulercomment}
\begin{eulerprompt}
>equ &= fs(K,R,P,n)=Kn; $&equ
\end{eulerprompt}
\begin{eulerformula}
\[
\frac{100\,\left(\left(\frac{P}{100}+1\right)^{n}-1\right)\,R}{P}+K
 \,\left(\frac{P}{100}+1\right)^{n}={\it Kn}
\]
\end{eulerformula}
\begin{eulercomment}
Biasanya rumus ini diberikan dalam bentuk

\end{eulercomment}
\begin{eulerformula}
\[
i=\frac{P}{100}
\]
\end{eulerformula}
\begin{eulerprompt}
>equ &= (equ with P=100*i); $&equ
\end{eulerprompt}
\begin{eulerformula}
\[
\frac{\left(\left(i+1\right)^{n}-1\right)\,R}{i}+\left(i+1\right)^{
 n}\,K={\it Kn}
\]
\end{eulerformula}
\begin{eulercomment}
Kita dapat mencari laju R secara simbolis.
\end{eulercomment}
\begin{eulerprompt}
>$&solve(equ,R)
\end{eulerprompt}
\begin{eulerformula}
\[
\left[ R=\frac{i\,{\it Kn}-i\,\left(i+1\right)^{n}\,K}{\left(i+1
 \right)^{n}-1} \right] 
\]
\end{eulerformula}
\begin{eulercomment}
Seperti yang dapat Anda lihat dari rumus, fungsi ini mengembalikan
kesalahan floating point untuk i=0. Namun, Euler memplotnya.

Tentu saja, kita memiliki limit berikut.
\end{eulercomment}
\begin{eulerprompt}
>$&limit(R(5000,0,x,10),x,0)
\end{eulerprompt}
\begin{eulerformula}
\[
\lim_{x\rightarrow 0}{R\left(5000 , 0 , x , 10\right)}
\]
\end{eulerformula}
\begin{eulercomment}
Jelas, tanpa bunga, kita harus membayar kembali 10 suku bunga sebesar
500.

Persamaan ini juga dapat diselesaikan untuk n. Akan terlihat lebih
bagus jika kita menerapkan beberapa penyederhanaan.
\end{eulercomment}
\begin{eulerprompt}
>fn &= solve(equ,n) | ratsimp; $&fn
\end{eulerprompt}
\begin{eulerformula}
\[
\left[ n=\frac{\log \left(\frac{R+i\,{\it Kn}}{R+i\,K}\right)}{
 \log \left(i+1\right)} \right] 
\]
\end{eulerformula}
\eulerheading{Contoh Soal Perhitungan Aljabar}
\begin{eulercomment}
Menjabarkan persamaan\\
\end{eulercomment}
\begin{eulerformula}
\[
(x-1)^2
\]
\end{eulerformula}
\begin{eulerprompt}
>$&expand((x-1)^2)
\end{eulerprompt}
\begin{eulerformula}
\[
x^2-2\,x+1
\]
\end{eulerformula}
\begin{eulercomment}
Menentukan nilai x dari sebuah persamaan\\
\end{eulercomment}
\begin{eulerformula}
\[
x^2-2x-3
\]
\end{eulerformula}
\begin{eulerprompt}
>$&solve(x^2+2*x-3)
\end{eulerprompt}
\begin{eulerformula}
\[
\left[ x=-3 , x=1 \right] 
\]
\end{eulerformula}
\begin{eulercomment}
Memfaktorkan persamaan\\
\end{eulercomment}
\begin{eulerformula}
\[
x^2-2x-3
\]
\end{eulerformula}
\begin{eulerprompt}
>$&factor(x^2+2*x-3)
\end{eulerprompt}
\begin{eulerformula}
\[
\left(x-1\right)\,\left(x+3\right)
\]
\end{eulerformula}
\begin{eulercomment}
Menentukan turunan dari persamaan\\
\end{eulercomment}
\begin{eulerformula}
\[
15t^2-2^t+7
\]
\end{eulerformula}
\begin{eulerprompt}
>$&diff(5*t^3-2*t+7,t)
\end{eulerprompt}
\begin{eulerformula}
\[
15\,t^2-2
\]
\end{eulerformula}
\begin{eulercomment}
Menentukan hasil dari\\
\end{eulercomment}
\begin{eulerformula}
\[
\int 3a^2-2a+2\ da
\]
\end{eulerformula}
\begin{eulerprompt}
>$&integrate(3*a^2-2*a+2,a)
\end{eulerprompt}
\begin{eulerformula}
\[
a^3-a^2+2\,a
\]
\end{eulerformula}
\begin{eulercomment}
Menentukan luas lingkaran dengan jari-jari 25
\end{eulercomment}
\begin{eulerprompt}
>L:=pi*r^2; r:=25; L
\end{eulerprompt}
\begin{euleroutput}
        12.57 
\end{euleroutput}
\begin{eulercomment}
Melakukan perhitungan pada bilangan kompleks
\end{eulercomment}
\begin{eulerprompt}
>(7i+8)+(12+3i)
\end{eulerprompt}
\begin{euleroutput}
             20.00+10.00i 
\end{euleroutput}
\begin{eulercomment}
Melakukan perhitungan menggunakan fungsi buatan sendiri yaitu\\
\end{eulercomment}
\begin{eulerformula}
\[
g(x)=2x^2+4x-2
\]
\end{eulerformula}
\begin{eulercomment}
kemudian disubtitusikan dengan nilai x=4
\end{eulercomment}
\begin{eulerprompt}
>function g(x)&=2*x^2+4*x-2; g(4)
\end{eulerprompt}
\begin{euleroutput}
        46.00 
\end{euleroutput}
\begin{eulercomment}
Menyelesaikan suatu sistem persamaan linear dua variabel\\
\end{eulercomment}
\begin{eulerformula}
\[
2x-y=7
\]
\end{eulerformula}
\begin{eulerformula}
\[
x+3y=14
\]
\end{eulerformula}
\begin{eulerprompt}
>$&solve([2*x-y=7,x+3*y=14],[x,y])
\end{eulerprompt}
\begin{eulerformula}
\[
\left[ \left[ x=5 , y=3 \right]  \right] 
\]
\end{eulerformula}
\begin{eulercomment}
Menyelesaikan suatu sistem pertidaksamaan dua variabel
\end{eulercomment}
\begin{eulerprompt}
>$&fourier_elim([3*x-4*y>12, x+5*y<=5],[x,y])
\end{eulerprompt}
\begin{eulerformula}
\[
\left[ x=5-5\,y , y<\frac{3}{19} \right] \lor \left[ \frac{4\,y}{3}
 +4<x , x<5-5\,y , y<\frac{3}{19} \right] 
\]
\end{eulerformula}
\begin{eulercomment}
Diberikan sebuah matriks berukuran 2x2\\
\end{eulercomment}
\begin{eulerformula}
\[
A = \begin{bmatrix} 1&2 \\ 3 & 12 \end{bmatrix}
\]
\end{eulerformula}
\begin{eulercomment}
kemudian tentukan:\\
AxA (perkalian produk)\\
3A (Perkalian dengan skalar)\\
determinan (A)\\
invers A\\
Buktikan bahwa elemen identitasnya adalah \\
\end{eulercomment}
\begin{eulerformula}
\[
I = \begin{bmatrix} 1&0 \\ 0&1 \end{bmatrix}
\]
\end{eulerformula}
\begin{eulerprompt}
>A=[1,2;3,12]; A.A
\end{eulerprompt}
\begin{euleroutput}
         7.00       26.00 
        39.00      150.00 
\end{euleroutput}
\begin{eulerprompt}
>3*A
\end{eulerprompt}
\begin{euleroutput}
         3.00        6.00 
         9.00       36.00 
\end{euleroutput}
\begin{eulerprompt}
>det(A)
\end{eulerprompt}
\begin{euleroutput}
         6.00 
\end{euleroutput}
\begin{eulerprompt}
>invA:=invert(A)
\end{eulerprompt}
\begin{euleroutput}
         2.00       -0.33 
        -0.50        0.17 
\end{euleroutput}
\begin{eulerprompt}
>A.invA
\end{eulerprompt}
\begin{euleroutput}
         1.00        0.00 
         0.00        1.00 
\end{euleroutput}
\eulerheading{Contoh soal aljabar pada masalah sehari-hari}
\begin{eulercomment}
Diberikan sebuah rumus untuk menghitung bunga tabungan yaitu\\
\end{eulercomment}
\begin{eulerformula}
\[
S = P\left[\frac{\left(1+{\frac{r}{12}}\right)^{12\cdot t}-1}{\frac{r}{12}}\right]
\]
\end{eulerformula}
\begin{eulercomment}
dengan S jumlah tabungan jika deposito P dollar dimasukkan setiap
bulan selama t tahun dalam rekening dengan suku bunga r
\end{eulercomment}
\begin{eulerprompt}
>function S(P,r,t)&=P*(((1+(r/12))^(12*t)-1)/(r/12));
\end{eulerprompt}
\begin{eulercomment}
James menyimpan \textdollar{}250 di rekening pensiun setiap bulan dimulai pada
usia 40 tahun. Jika investasi tersebut menghasilkan bunga 5\%,
dibungakan setiap bulan, berapa banyak yang akan terkumpul di rekening
tersebut ketika ia pensiun 27 tahun kemudian?
\end{eulercomment}
\begin{eulerprompt}
>S(250,0.05,27)
\end{eulerprompt}
\begin{euleroutput}
    170797.30 
\end{euleroutput}
\begin{eulercomment}
Pada perintah diatas saya membuat suatu fungsi untuk menampung rumus
perhitungannya sehingga jika saya ingin menghitung sesuai dengan
keadaan yang ada, saya tinggal menyesuaikan dengan kebutuhannya

Kayla menyimpan \textdollar{}100 di rekening pensiun setiap bulan dimulai pada
usia 25 tahun. Jika investasi tersebut menghasilkan bunga 4\%,
dibungakan setiap bulan, berapa banyak yang akan terkumpul di rekening
tersebut ketika ia pensiun pada usia 65 tahun?
\end{eulercomment}
\begin{eulerprompt}
>S(100,0.04,40)
\end{eulerprompt}
\begin{euleroutput}
    118196.13 
\end{euleroutput}
\end{eulernotebook}
\end{document}
